%%=============================================================================
%% Tooling
%%=============================================================================

\chapter{\IfLanguageName{dutch}{Tooling}{Tooling}}
\label{ch:tooling}
Een besturingssysteem is een interface tussen de gebruiker en de hardwarecomponenten. Het voert verschillende taken uit. Een besturingssysteem biedt de gebruiker een Graphical User Interface (GUI) of Command Line Interface (CLI) aan om taken uit te voeren. Sommige besturingssystemen bieden alleen een CLI of GUI aan, terwijl andere zowel GUI als CLI aanbieden. Een GUI bestaat uit bedieningselementen of widgets om met de computer te communiceren. Aan de andere kant moet de gebruiker bij gebruik van de CLI opdrachten invoeren om de taken uit te voeren. Over het algemeen is GUI gebruiksvriendelijker, maar de uitvoeringssnelheid is hoger in CLI.

Een CLI opdracht wordt dus uitgevoerd aan de hand van een commando. 

Nu is het mogelijk om een aantal extra commando’s te installeren die deel uitmaken van een overkoepelende tool. 

Dit hoofdstuk onderzoek de verschillende commando’s van de Flutter tool en vergelijkt deze met de verschillende Android tools en hun gelijkende commando’s.

\section{Android}
De Android SDK bestaat uit meerdere tools die nodig zijn voor app-ontwikkeling. De tools kunnen geinstalleerd en bijgewerkt worden met SDK Manager van Android Studio of de command line tool ‘sdkmanager’. Alle tools worden gedownload naar de Android SDK-map. Aangezien de aard van het onderzoek een vergelijkende studie is werden hier enkel de commando’s aangehaald met een Flutter tegengestelde. Het aanhalen van alle commando’s voor elke tool zou te veel en niet interessant zijn.

\section{Flutter}
De Flutter command-line tool is hoe ontwikkelaars (en IDE’s aangestuurd door ontwikkelaars) omgaan met Flutter. 

Een Software Development Kit (SDK) is een verzameling softwareontwikkelingstools in één installeerbaar pakket. Ze vergemakkelijken het maken van applicaties door een compiler, debugger en misschien een softwareframework te bevatten. Ze zijn normaal gesproken specifiek voor een combinatie van hardwareplatform en besturingssysteem. Een SDK voorziet een set tools, bibliotheken, relevante documentatie, codevoorbeelden, processen en/of handleidingen waarmee ontwikkelaars softwaretoepassingen op een specifiek platform kunnen maken. 

Voor het kunnen werken met Flutter moet dus eerst de Flutter SDK geinstalleerd worden. Deze SDK bevat onder andere een aantal commando’s die kunnen uitgevoerd worden in de terminal. Flutter commando’s worden voorafgegaan door het flutter keyword. Zo weet het besturingssysteem dat een commando wordt gebruikt van de flutter tool.

Hieronder staat een lijst met de mogelijke flutter commando’s.

\section{Conclusie}