%%=============================================================================
%% Tooling
%%=============================================================================

\chapter{\IfLanguageName{dutch}{Tooling}{Tooling}}
\label{ch:tooling}
Een besturingssysteem is een interface tussen de gebruiker en de hardwarecomponenten. Het voert verschillende taken uit. Een besturingssysteem biedt de gebruiker een Graphical User Interface (GUI) of Command Line Interface (CLI) aan om taken uit te voeren. Sommige besturingssystemen bieden alleen een CLI of GUI aan, terwijl andere zowel GUI als CLI aanbieden. Een GUI bestaat uit bedieningselementen of widgets om met de computer te communiceren. Aan de andere kant moet de gebruiker bij gebruik van de CLI opdrachten invoeren om de taken uit te voeren. Over het algemeen is GUI gebruiksvriendelijker, maar de uitvoeringssnelheid is hoger in CLI.

Een CLI opdracht wordt dus uitgevoerd aan de hand van een commando. 

Nu is het mogelijk om een aantal extra commando’s te installeren die deel uitmaken van een overkoepelende tool. 

Dit hoofdstuk onderzoek de verschillende commando’s van de Flutter tool en vergelijkt deze met de verschillende Android tools en hun gelijkende commando’s. \newpage

\section{Opzet}
Een Software Development Kit (SDK) is een verzameling softwareontwikkelingstools in één installeerbaar pakket. Ze vergemakkelijken het maken van applicaties door een compiler, debugger en misschien een softwareframework te bevatten. Ze zijn normaal gesproken specifiek voor een combinatie van hardwareplatform en besturingssysteem. Een SDK voorziet een set tools, bibliotheken, relevante documentatie, codevoorbeelden, processen en/of handleidingen waarmee ontwikkelaars softwaretoepassingen op een specifiek platform kunnen maken. 



\section{Android}
De Android SDK bestaat uit meerdere tools die nodig zijn voor app-ontwikkeling. De tools kunnen geinstalleerd en bijgewerkt worden met SDK Manager van Android Studio of de command line tool ‘sdkmanager’. Alle tools worden gedownload naar de Android SDK-map. Aangezien de aard van het onderzoek een vergelijkende studie is werden hier enkel de commando’s aangehaald met een Flutter tegengestelde. Het aanhalen van alle commando’s voor elke tool zou te veel en niet interessant zijn.

\begin{tabular}{ |p{3cm}|p{11cm}|  }
    \hline
    \multicolumn{2}{|c|}{\textbf{Android SDK Command-Line Tools}} \\
    \multicolumn{2}{|c|}{(android\_sdk/cmdline-tools/version/bin/)} \\
    \hline
    apkanalyzer & Geeft inzicht in de opbouw van de APK nadat het build proces klaar is.\\
    \hline
    avdmanager & Staat het toe om Android Virtual Devices (AVD's) te maken en beheren vanaf de command line.\\
    \hline
    lint & Scant de code en helpt bij het identificeren en corrigeren van problemen met de structurele kwaliteit.\\
    \hline
    sdkmanager & Staat het toe om Android SDK packages te installeren, te updaten en te verwijderen.\\
    \hline
\end{tabular}

\begin{tabular}{ |p{3cm}|p{11cm}|  }
    \hline
    \multicolumn{2}{|c|}{\textbf{Android SDK Build Tools}} \\
    \multicolumn{2}{|c|}{(android\_sdk/build-tools/version/)} \\
    \hline
    \multicolumn{2}{|c|}{Deze package is vereist om Android apps te bouwen. De meeste tools in deze} \\
    \multicolumn{2}{|c|}{package worden aangeroepen door de build-tools.} \\
    \hline
    aapt2 & Parses, indexes en compiles Android resources in een binair formaat dat is geoptimaliseerd voor het Android-platform en verpakt de gecompileerde resources in een enkele uitvoer.\\
    \hline
    apksigner & Ondertekend APK's en controleert of de APK handtekeningen met succes zullen worden geverifieerd op alle platform versies die een bepaalde APK ondersteunt.\\
    \hline
    zipalign & Optimaliseert APK-bestanden door ervoor te zorgen dat alle niet-gecomprimeerde gegevens beginnen met een bepaalde uitlijning ten opzichte van het begin van het bestand.\\
    \hline
\end{tabular}

\begin{tabular}{ |p{3cm}|p{11cm}|  }
    \hline
    \multicolumn{2}{|c|}{\textbf{Android Emulator }} \\
    \multicolumn{2}{|c|}{(android\_sdk/emulator/)} \\
    \hline
    \multicolumn{2}{|c|}{Deze package is vereist om de Android Emulator te gebruiken.} \\
    \hline
    emulator & Een op QEMU gebaseerde apparaat emulatie tool die kan gebruikt worden om de applicaties te debuggen en te testen in een echte Android-runtime omgeving.\\
    \hline
\end{tabular}

\begin{tabular}{ |p{3cm}|p{11cm}|  }
    \hline
    \multicolumn{2}{|c|}{\textbf{Android SDK Platform Tools}} \\
    \multicolumn{2}{|c|}{(android\_sdk/build-tools/version)} \\
    \hline
    \multicolumn{2}{|c|}{Deze tools worden bijgewerkt voor elke nieuwe versie van het Android-platform om} \\
    \multicolumn{2}{|c|}{nieuwe functies te ondersteunen (en soms om de tools te verbeteren) en elke update} \\
    \multicolumn{2}{|c|}{is achterwaarts compatibel met oudere platformversies.} \\
    \hline
    adb & Android Debug Bridge (adb) is een veelzijdige tool waarmee de status van een emulatori of Android-apparaat kan beheert worden. Het kan ook gebruikt worden om een ​​APK op een apparaat te installeren.\\
    \hline
    logcat & Is een tool die via adb wordt aangeroepen om app- en systeem logs te bekijken.\\
    \hline
\end{tabular}

\section{Flutter}
De Flutter command-line tool is hoe ontwikkelaars (en IDE’s aangestuurd door ontwikkelaars) omgaan met Flutter. 

Voor het kunnen werken met Flutter moet dus eerst de Flutter SDK geïnstalleerd worden. Deze SDK bevat onder andere een aantal commando’s die kunnen uitgevoerd worden in de terminal. Flutter commando’s worden voorafgegaan door het flutter keyword. Zo weet het besturingssysteem dat een commando wordt gebruikt van de flutter tool.

De dart-tool is een command line interface voor de Dart SDK. De tool is beschikbaar, ongeacht hoe de Dart SDK gedownload werd (expliciete downloadt of alleen de Flutter SDK downloadt). Indien het mogelijk is om de flutter tool te gebruiken is dit het betere alternatief.

Hieronder staat een lijst met de mogelijke flutter en dart commando’s.

\begin{tabular}{ |p{3cm}|p{11cm}|  }
    \hline
    \multicolumn{2}{|c|}{\textbf{Flutter command-line tool}} \\
    \multicolumn{2}{|c|}{(android\_sdk/build-tools/version)} \\
    \hline
    \multicolumn{2}{|c|}{Deze commando's worden als volgt opgebouwd: flutter <commando> <optie> <parameter> } \\
    \hline
    analyze & Analyseert de Dart broncode van het project. Gebruik dit in plaats van dartanalyzer.\\
    \hline
    assemble & Verzamel en maak Flutter resources.\\
    \hline
    attach & Verbind het toestel met een draaiende applicatie\\
    \hline
    build & Flutter build commando's\\
    \hline
    channel & Toont een lijst van mogelijke flutter channels, of verander van channel.\\
    \hline
    config & Configureer Flutter instellingen. Om een ​​instelling te verwijderen, configureer deze als een lege string.\\
    \hline
    create & Maakt een nieuw project aan.\\
    \hline
    devices & Maakt een lijst van alle aangesloten apparaten.\\
    \hline
    doctor & Toont informatie over de geïnstalleerde tooling.\\
    \hline
    downgrade & Downgrade Flutter naar de laatste actieve versie voor de huidige channel.\\
    \hline
    drive & Voert Flutter Driver-tests uit voor het huidige project.\\
    \hline
    emulators & Lijst, start en maak emulators.\\
    \hline
    format & Formateert Flutter broncode. Gebruik dit in plaats van dartfmt.\\
    \hline
    gen-l10n & Genereert localizations voor het Flutter-project.\\
    \hline
    install & Installeer een Flutter app op een aangesloten apparaat.\\
    \hline
    logs & Toon log uitvoer voor draaiende Flutter apps.\\
    \hline
    pub & Werkt met packages.\\
    \hline
    run & Voert een Flutter applicatie uit\\
    \hline
    symbolize & Symboliseer een stacktracering van de door AOT gecompileerde Flutter applicatie.\\
    \hline
    test & Voert tests uit in de huidige package.\\
    \hline
    upgrade & Upgrade Flutter.\\
    \hline
\end{tabular}

\begin{tabular}{ |p{3cm}|p{11cm}|  }
    \hline
    \multicolumn{2}{|c|}{\textbf{Dart command-line tools}} \\
    \multicolumn{2}{|c|}{(bin/dart)} \\
    \hline
    \multicolumn{2}{|c|}{Deze commando's worden als volgt opgebouwd: dart <commando> <optie> <parameter> } \\
    \hline
    analyze & Analyseert de Dart broncode van het project.\\
    \hline
    compile & Compileert Dart naar verschillende formaten.\\
    \hline
    create & Maakt een nieuw project\\
    \hline
    fix & Past geautomatiseerde oplossingen toe op de Dart broncode.\\
    \hline
    format & Formatteert Dart broncode.\\
    \hline
    migrate & Ondersteunt migratie naar null-safety.\\
    \hline
    pub & Werkt met packages.\\
    \hline
    run & Voert een Dart progamma uit\\
    \hline
    test & Voert tests uit in de huidige package\\
    \hline
    (leeg) & Voert een Dart-programma uit; identiek aan de reeds bestaande Dart VM-opdracht.\\
    \hline
\end{tabular}

\section{Conclusie}
Aangezien Flutter geen toegewijde IDE heeft moeten meer commando's uitgevoerd worden in de terminal. 