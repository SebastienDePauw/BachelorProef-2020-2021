%%=============================================================================
%% Conclusie
%%=============================================================================

\chapter{Conclusie}
\label{ch:conclusie}

% Trek een duidelijke conclusie, in de vorm van een antwoord op de
% onderzoeksvra(a)g(en). Wat was jouw bijdrage aan het onderzoeksdomein en
% hoe biedt dit meerwaarde aan het vakgebied/doelgroep? 
% Reflecteer kritisch over het resultaat. In Engelse teksten wordt deze sectie
% ``Discussion'' genoemd. Had je deze uitkomst verwacht? Zijn er zaken die nog
% niet duidelijk zijn?
% Heeft het onderzoek geleid tot nieuwe vragen die uitnodigen tot verder 
%onderzoek?

Voor elk van voorgaande onderzoekscriteria kan gezegd worden dat Flutter een goed, als niet beter alternatief biedt op native Android. Het lijkt de betere optie te zijn voor kleinschalige apps, minder boilerplate code en een minder steile leercurve. Het grootste voordeel dat Flutter biedt tegenover Android is de enkele codebase die kan gecompileerd worden naar Android- maar ook iOS- en web applicaties. Dit kunnen de meeste cross-platform systemen en is dus niet eigen aan Flutter. Een ander Flutter specifiek voordeel is de hot reload functionaliteit die de ontwikkelcyclus zoveel sneller maakt. Nog een voordeel van Flutter is de overzichtelijke documentatie en een groepering van alle verschillende packages op één site. Al kan het moeilijk zijn om specifieke goed ondersteunde packages te vinden. Een aantal nadelen van Flutter zijn het verloren gaan van de native look en feel. Maar ook de nieuwe iOS en Android features die pas later worden toegevoegd.

Het is moeilijk te bepalen of Flutter al matuur genoeg is om te gebruiken in de professionele sector. Het is belangrijk om rekening te houden met de omvang van de applicaties. De apps ontwikkeld voor dit onderzoek bevatten geen complexe features. De resultaten en conclusies van dit onderzoek slaan dus alleen op kleine Flutter apps. Grootschalige applicaties met complexe features zouden een andere kijk op het systeem kunnen geven. Het maken van een complexe applicatie viel buiten de scope van het onderzoek.  Wel kan gezegd worden dat voor kleinschalige applicaties met simpele features Flutter een goed alternatief biedt op Android. De snelle vooruitgang, de grote gebruikersbasis en de vele updates binnen het platform geven ook teken van een positieve toekomst. 

Flutter was zeer toegankelijk voor nieuwe developers. Voor moeilijke concepten zoals asynchrone taken voorziet flutter simpele oplossingen die gemakkelijk aan te leren zijn. Indien iets niet duidelijk is kan ook altijd terug gevallen worden op de uitgebreide documentatie. De diepe nesting van Widgets kan in het begin ingewikkeld lijken maar is een concept dat snel duidelijk wordt. Het kan voor nieuwe ontwikkelaars soms beangstigend zijn om via de terminal te werken. Kleine simpele Flutter apps kunnen geschreven worden zonder kennis van Android of iOS, wat veel tijd uitspaart. Echter vergt Flutter wel kennis over onderliggende frameworks wanneer men complexere Flutter apps begint te ontwikkelen. Daarom is het niet aangeraden om zonder voorkennis in applicatieontwikkeling, Flutter apps te ontwikkelen.

Ook zijn bepaalde delen van het onderzoek open voor discussie. Zowel de resultaten van CPU gebruik als die van opstartsnelheid worden in vraag gesteld. Zo werden niet alle omgevingsvariabelen weg gehaald voor en tijdens het onderzoek. Deze conclusie ligt in lijn met de hypotheses op de onderzoeksvragen. De literatuurstudie gaf duidelijk inzicht in het Flutter framework, hieruit kon een objectief beeld gevormd worden over de werking en de voor- en nadelen van de Flutter stack. 

Dit onderzoek spitste zich toe op het vergelijken van native Android apps met Flutter apps, echter viel een vergelijking met iOS buiten deze scope. In verder onderzoek zou deze vergelijking kunnen gemaakt worden. Hieruit kan dan een volwaardig beeld worden geschetst van de huidige stand van het Flutter framework tegenover native ontwikkeling. Daarnaast kan verder onderzoek gedaan worden naar Flutter app performantie. Deze metric zal normaal blijven verbeteren naarmate het platform groeit. Om dit te kunnen toetsen is toekomstig onderzoek een must. Door deze performantie resultaten te vergelijken over de jaren heen kan gekeken worden naar de groei van Flutter. Als laatste kan de schaalbaarheid van Flutter apps een interessant onderzoeksdomein vormen voor de bedrijfswereld. Professionele apps bevatten meestal veel complexe features die Android op een goede manier integreert. Indien Flutter meer tractie wenst te krijgen moet het een uitwerking of alternatief kunnen bieden voor elke van deze features.