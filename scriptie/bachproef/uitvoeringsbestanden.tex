\section{De grootte van de uitvoeringsbestanden}
\label{sec:uitvoeringsbestanden}
Het eerste performantie aspect dat onderzocht werd was de grootte van het uitvoeringsbestand. Het doel bij applicatieontwikkeling is om een uitvoeringsbestand zo klein mogelijk te houden, zodat de gebruiker geheugen kan besparen op zijn toestel. Het is immers zo dat de prijs van high-end smartphones vandaag de dag vrij steil is. De consument kan bij het aanschaffen van een nieuw toestel vaak kiezen voor een bepaalde hoeveelheid opslag. Wanneer voor een toestel gekozen wordt met een vrij beperkte opslagcapaciteit en de gebruiker een aantal grote apps installeert, zal de opslagcapaciteit van het toestel snel volledig ingenomen zijn. Dit kan tot gevolg hebben dat de gebruiker bepaalde grotere applicaties zal gaan verwijderen om ruimte te creëren. Dit is één van de redenen waarom om een app zo klein mogelijk te houden en ondertussen de hoogste performantie te garanderen. 

Een andere reden is de maximale toegestane grootte van de uitvoeringsbestanden opgelegd door de verschillende app stores. Om een applicatie op de Google Play Store te kunnen beschikbaar stellen, moet deze voldoen aan meerdere eisen. Een van deze eisen is dat het uitvoeringsbestand van de app maximaal 100MB bedraagt. Het doel van een applicatie beschikbaar te stellen op de Play Store is immers om een zo groot mogelijk doelpubliek aan te spreken. Wanneer echter niet aan de eisen van de Play Store voldaan wordt, zal dit doelpubliek niet bereikt worden. In dit hoofdstuk zal gekeken worden naar de grootte van de uitvoeringsbestanden van respectievelijk de Flutter applicatie alsook de Android applicatie.

\subsection{Opzet}
Een uitvoeringsbestand, ook wel gekend als een executable, van een Android app kan verschillende bestandsformaten zijn. De twee meest gebruikte formaten zijn Android Package (APK) en App Bundle. Deze bestanden worden gebruikt om een Android applicatie uit te voeren. Voor dit onderzoek werd gefocust op het APK bestandsformaat. Een APK is een map of verzameling van bestanden. Al deze bestanden samen zorgen ervoor dat een applicatie kan worden uitgevoerd. Zo werden voor dit onderzoek twee APK's gegenereerd van de Hello world applicaties. De eerste vanuit de native Android codebase en de tweede van de Flutter codebase. Voor de meest bruikbare resultaten werd gekozen om voor de applicaties een release build te maken. Elke build is een verzameling van een aantal regels die worden toegepast op de code wanneer deze compileert. De release build is een verzameling van regels die de app zo compact mogelijk maakt. Dit gebeurt door het opruimen en optimaliseren van code. De resultaten van het onderzoek werden bekomen gebruik makend van de apkanalyzer, een ingebouwde Android Studio tool die de APK ontleed en de grootte van elk van deze delen weergeeft. 

Tabel \ref{table:maatstafUitvoeringsbestand} biedt inzicht in de gebruikte maatstaf voor dit deel van het onderzoek.
\begin{table}
    \begin{center}
        \caption{Gebruikte maatstaf voor de grootte van de uitvoeringsbestanden}
        \label{table:maatstafUitvoeringsbestand}
        \begin{tabular}{ |l|c|c| }
            \hline
            Bit & / & Binair getal, 1 of 0\\
            \hline
            Byte & B & 8 Bit \\ 
            \hline
            Kilobyte & KB &  \[10^{3}\]B\\ 
            \hline
            Megabyte & MB & \[10^{6}\]B \\ 
            \hline
        \end{tabular}
    \end{center}
\end{table}

\newpage

\textbf{Android}\\
In Android wordt een release build gemaakt aan de hand van volgende lijnen code in het app build.gradle bestand \ref{lst:label}.
\begin{lstlisting} [caption={Android build.gradle (app)},  label={lst:label}]
release {
    minifyEnabled true
    shrinkResources true
    proguardFiles getDefaultProguardFile
    ('proguard-android-optimize.txt'), 'proguard-rules.pro'
}
\end{lstlisting} 

\textbf{Flutter}\\
Voor een ideale Flutter release build wordt best gebruik gemaakt van onderstaande terminal commando’s.\\
\textit{flutter clean}\\
\textit{flutter build apk --split-per-abi}\\
Het flutter clean commando zal het project kleiner maken door het verwijderen van de build en .dart-tool mappen. Het commando daarna maakt een release APK per ABI. De APK’s worden gesplitst per ABI omdat dit het uitvoeringsbestand aanzienlijk verkleint. Verdere uitleg over ABI kan terug gevonden worden in hoofdstuk \ref{ch:appendix} Appendix.

Als laatste werd een opsomming gemaakt van de grootte van Flutter uitvoeringsbestanden over tijd. Dit moet een inzicht bieden in de evolutie van Flutter.

\subsection{Resultaten}
\textbf{Android}\\
Een Android app heeft geen minimum grootte voor het uitvoeringsbestand. Echter als de app op de Play Store dient gezet te worden moet deze minimum 7KB bedragen. Wanneer een nieuw Native Kotlin Android project wordt opgezet met één activity, zal dit automatisch een Hello World applicatie zijn. Als eerste stap in het onderzoek werd gekeken naar de omvang van deze APK. Volgens apkanalyzer was deze APK 3.2 MB in omvang en 2.6 MB voor de download. Vervolgens werd gekeken naar een verkleinde versie van de codebase waaruit de testing directories verwijderd werden samen met de bijhorende dependencies. Vervolgens werd minifyEnabled als ook shrinkResources op true gezet. Dit leidt tot een APK van 1.6 MB met een downloadgrootte van 1010.1KB.
Voor een derde en finale versie van de APK werd de codebase nog kleiner gemaakt, met als gevolg dat niet meer werd gewerkt volgens de Android best practices. Tabel \ref{table:androidUitvoerinsbestanden} bevat de resultaten van dit onderzoek. De code voor deze Hello world applicatie kan terug gevonden worden op GitHub repository \autocite{DePauw2021}.

\begin{table}
    \begin{center}
        \caption{Grootte van de uitvoeringsbestanden van de Hello world app in Android \autocite{DePauw2021}}
        \label{table:androidUitvoerinsbestanden}
    \begin{tabular}{ | l | m{3cm} | m{3cm} | }
        \hline
        & \textbf{APK size} & \textbf{Download Size}\\
        \hline
        META-INF & 3 KB & 3 KB\\ 
        \hline
        res & 123,7 Kb & 117 KB\\ 
        \hline
        AndroidManifest,xml & 721 B & 721 B\\ 
        \hline
        classes.dex & 293,9 KB & 293,1 KB\\ 
        \hline
        Resources.arsc & 222,6 KB & 51,7 KB\\ 
        \hline
        Kotlin & 9,1 KB & 9 KB\\ 
        \hline
        \hline
        \textbf{Total} & \textbf{693,2 KB} & \textbf{476,1 KB} \\ 
        \hline
    \end{tabular}
\end{center}
\end{table}

\textbf{Flutter}\\
Flutter maakt gebruik van de Flutter Engine (zie hoofdstuk \ref{ch:appendix} Appendix). Deze engine maakt deel uit van de APK en is nodig voor het gebruiken van een Flutter app. Deze engine bevat het gehele framework en is daarom een aantal megabyte groot. Dit maakt de Flutter APK al direct meerdere megabytes in omvang. Voor het eerste deel van het onderzoek werd gekeken naar een standaard versie van de Hello world template. Hierbij bevatte de main.dart file 26 lijnen code. De niet gesplitste ABI versie was 15.5 MB groot terwijl de gesplitste APK 5.1 MB bedroeg voor de gewone ARM en 5.5 MB voor de ARM64. Vervolgens werd de applicatie herschreven met als doel een zo klein mogelijke APK te genereren. Hierbij was het main.dart bestand 12 lijnen code na het uitvoeren van een code format. De APK versie waarbij niet gesplitst werd per ABI was 14 MB groot. De resultaten van de APK gesplitst op ABI staan vermeld in tabel \ref{table:flutterUitvoeringsbestanden}. De code voor dit onderzoek kan terug gevonden worden op de GitHub repository \autocite{DePauw2021}.

\begin{table}
    \begin{center}
        \caption{Grootte van de uitvoeringsbestanden van de Hello world app in Flutter \autocite{DePauw2021}}
        \label{table:flutterUitvoeringsbestanden}
        \begin{tabular}{ | l | m{2.5cm} | m{2.5cm} | m{2.5cm} | m{2.5cm} | }
            \hline
            & \multicolumn{2}{|c|}{\textbf{ARM}} & \multicolumn{2}{|c|}{\textbf{ARM-64}}\\
            \hline
            & \textbf{APK Size} & \textbf{Download size} & \textbf{Apk Size} & \textbf{Download Size}\\ 
            \hline
            lib & 4,2 MB & 4,2 MB & 4,7 MB & 4,6 MB\\ 
            \hline
            assets & 183,4 KB & 183 KB & 183,4 KB & 183 KB\\ 
            \hline
            META-INF & 7,9 KB & 7,6 KB & 7,9 KB & 7,6 KB\\ 
            \hline
            res & 6,2 KB & 5,9 KB & 6,2 KB & 5,9 KB\\ 
            \hline
            AndroidManifest.xml & 1021 B & 1021 B & 1022 B & 1022 B\\ 
            \hline
            classes.dex & 121,7 KB & 121,3 KB & 121,7 KB & 121,3 KB\\ 
            \hline
            Resources.arsc & 22,6 KB & 3,8 KB & 22,6 KB & 3,8 KB\\ 
            \hline
            Kotlin & 9,7 KB & 9,7 KB & 9,7 KB & 9,7 KB\\ 
            \hline
            \hline
            \textbf{Total} &  \textbf{4,6 MB} &  \textbf{4,5 MB} &  \textbf{5 MB} &  \textbf{5 MB}\\ 
            \hline
        \end{tabular}
    \end{center}
\end{table}

Op de Flutter site \autocite{Flutter} staat een korte omschrijving van de minimale downloadgrootte gemeten van een Flutter app (geen materiële componenten, slechts een enkele Center-widget, gebouwd met flutter build apk --split-per-abi), gebundeld en gecomprimeerd als een release APK. Deze minimale downloadgrootte werd in 2018 verschillende malen verkleint en opnieuw gemeten. Tabel \ref{table:flutterUitvoeringsbestandenOverTijd} bundelt de resultaten van deze metingen.

\begin{table}
    \begin{center}
        \caption{Grootte van de uitvoeringsbestanden in Flutter over tijd}
        \label{table:flutterUitvoeringsbestandenOverTijd}
        \begin{tabular}{ |l|c| }
            \hline
            \textbf{Tijd} & \textbf{Download Size}\\
            \hline
            Maart 2018 & 4,06 MB\\ 
            \hline
            Augustus 2018 & 4,20 MB\\ 
            \hline
            Begin Oktober 2018 & 4,28 MB\\ 
            \hline
            Eind Oktober 2018 & 4,48 MB\\ 
            \hline
            November 2018 & 4,70 MB\\ 
            \hline
            December 2018 & 6,70 MB\\ 
            \hline
        \end{tabular}
    \end{center}
\end{table}