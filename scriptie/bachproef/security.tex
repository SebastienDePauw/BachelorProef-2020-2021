%%=============================================================================
%% Security
%%=============================================================================


\chapter{\IfLanguageName{dutch}{App veiligheid}{App Security}}
\label{ch:security}
Security of beveiliging is vandaag de dag meer dan enkel een hot topic. Het is een nood en het moet een garantie zijn die aangeboden wordt aan de gebruiker. Doorheen de gehele applicatie is security een belangrijk aspect. Daarom is app beveiliging één van de grootste zorgen voor ontwikkelaars.
Door een applicatie op een veilige manier te bouwen helpt dit de gebruiker meer vertrouwen te plaatsen in deze applicatie en wordt de apparaat integriteit behouden.

In dit hoofdstuk worden een aantal best practices aangehaald. Het is niet mogelijk om ze allemaal te noemen, dus werd een lijst gemaakt van de belangrijkste. Hierbij wordt ook vermeld hoe deze best practices geïmplementeerd worden op zowel native Android als Flutter apps.\newpage

\section{Best-practices}
\textbf{Bewaar gegevens op een veilige manier}\\
De persoonlijke informatie van de veilige opslaggebruiker mag alleen op het apparaat blijven staan als dat nodig is, en alleen toegankelijk worden gemaakt voor geautoriseerde gebruikers en processen. API-toegangstokens moeten bijvoorbeeld veilig op het apparaat worden opgeslagen. Android Keystore biedt de mogelijkheid om cryptografische sleutels in een container op te slaan, waardoor het moeilijker wordt voor ongeautoriseerde toegang.

Ter vergelijking: de plug-in flutter\_secure\_storage biedt veilige opslag via Android KeyStore. Voor een permanente opslag van eenvoudige gegevens is de plugin shared\_preferences beschikbaar. Echter, citerend uit de officiële documentatie, kan geen van beide platformen garanderen dat schrijfbewerkingen na terugkeer op schijf blijven staan; daarom mag de plug-in niet worden gebruikt voor het opslaan van kritieke gegevens. Ontwikkelaars kunnen besluiten om Flutter-platformkanalen te gebruiken om platform specifieke API's aan te roepen voor hun behoeften.

\textbf{Houd SDK, services en dependencies up-to-date}\\
De meeste apps gebruiken externe bibliotheken en apparaat informatie om gespecialiseerde taken uit te voeren. Door de afhankelijkheden van de app up-to-date te houden, worden deze communicatiepunten veiliger gemaakt.

\textbf{Vraag alleen persmissie voor wat je nodig hebt}\\
Altijd om toestemming vragen wanneer het nodig is. Vraag geen toestemming vanaf het begin van de app en vraag alleen naar wat nodig is. Sommige apps kunnen bijvoorbeeld vragen om toegang tot de GPS-locatie, hoewel geen van de functies dit vereist. Dit is een slechte manier van werken voor twee redenen. Ten eerste worden onnodige privégegevens gegevens verzamelt. Ten tweede kunnen hackers mogelijk ook toegang krijgen tot deze gegevens als deze niet correct worden geïmplementeerd.

In Android worden permissies gevraagd aan de hand van een klein stuk code dat onder andere checkt of de permissie al is goedgekeurd of niet. Daarbij moet elke permissie in het manifest van de app worden vermeld.

In Flutter kunnen permissies worden gevraagd aan de hand van method channels of de permission\_handler plugin.

\textbf{Preventie van snapshots op de achtergrond}\\
Apps die financiële informatie presenteren of betalingsfunctionaliteiten bieden, vereisen een hoger niveau van gegevensprivacy. Wanneer een app naar de achtergrond gaat, maakt het besturingssysteem een momentopname van de laatst zichtbare status om in de taakwisselaar te presenteren. Het is daarom gewenst om te voorkomen dat rekeningsaldi en betalingsgegevens worden vastgelegd door snapshots op de achtergrond.

Voor Android behandelt de FLAG\_SECURE-vlag de inhoud van het venster als veilig, waardoor het niet in schermafbeeldingen verschijnt.

Omdat het voorkomen van snapshots op de achtergrond nauw verbonden is met de levenscyclus van een applicatie, biedt Flutter geen plug-in voor het voorkomen van snapshots. Om dit te bereiken moeten ontwikkelaars vertrouwen op native API's.

\textbf{Detectie van gejailbreakte en geroote apparaten}\\
Bij gejailbreakte apparaten worden hun ingebouwde beveiligingsmaatregelen ondermijnd, wat kwetsbaarheden introduceert en de persoonlijke informatie en inloggegevens van de gebruiker in gevaar kan brengen. Apps die persoonlijke informatie verwerken, mogen niet draaien op gejailbreakte apparaten; jailbreak-detectie is dus verplicht voor veel consumenten-apps. 

Jailbreak-detectie omvat het scannen op verdachte mappen, bestanden en codefragmenten, wat platformspecifiek is. Flutter's SDK ondersteunt deze functie niet, maar bibliotheken van derden die native code inpakken om taken uit te voeren zoals flutter\_jailbreak\_detection en root\_checker zijn beschikbaar. Toch zou het, in vergelijking met het gebruik van bibliotheken van derden, voor meer flexibiliteit en vertrouwen beter zijn om rechtstreeks het native framework te implementeren.

\textbf{Code verduistering}\\
Statische strings en tekstbestanden in een app package zijn relatief eenvoudig te bemachtigen en kunnen gevoelige informatie bevatten, zoals een API-sleutel. Een goed softwareontwerp minimaliseert de kans op het lekken van gevoelige informatie. Mocht het onvermijdelijk worden om informatie op een dergelijke manier te verwerken in de app, dan moet code verduistering worden gebruikt om de informatie door elkaar te gooien. 

Code verduistering is een proces dat een uitvoerbaar bestand wijzigt of functie- en klasse namen verbergt, waardoor het moeilijk is om reverse-engineering toe te passen. Het beschermt het intellectuele eigendom van de klant en voorkomt ongeautoriseerde toegang, diefstal van inloggegevens of beveiligingsproblemen. Meestal ontwikkeld met native build-tools zoals het verwijderen van debugger-symbolen, code-optimalisatie en in-code verduisteringstechnieken.

Dart ondersteunt codeverduistering, maar is nog niet grondig getest. Flutter verkleint de Androidhosts niet. Daarom is het de verantwoordelijkheid van de ontwikkelaar om indien nodig native Android applicaties te obfusceren. Android ProGuard is in dit geval een handig hulpmiddel.

\textbf{Encryptie-API}\\
Afgezien van direct beschikbare API's voor veilige verbinding en opslag, willen ontwikkelaars soms codering op de applicatielaag toepassen om de gegevensbeveiliging te verbeteren.
Aan de Flutter-kant bieden de crypto- en versleutelbibliotheken van Dart een reeks cryptografische hashfuncties met basisversleutelingsfuncties. Hoewel ze niet zo uitgebreid zijn als native Android-frameworks, moeten deze bibliotheken voldoen aan de basisvereisten voor gegevenscodering.

\textbf{Verbindingsbeveiliging}\\
Apps die gegevens via internet verzenden en ontvangen, moeten gevoelige informatie beschermen tegen afluisteren en onbevoegde toegang. HTTPS, bekend als HTTP over TLS (Transport Layer Security), wordt vaak gebruikt om mobiel app-verkeer over uit te voeren vanwege de voordelen van gegevenscodering en authenticatie. Naast standaard TLS-configuraties kunnen functies zoals het vastzetten van certificaten en wederzijdse authenticatie de verbindingsbeveiliging verder verbeteren. Slecht geconfigureerde TLS-beveiligingsparameters, kunnen de verbinding kwetsbaar maken voor man-in-the-middle-aanvallen.

De functie Network Security Configuration op Android stelt ontwikkelaars in staat zich af te melden voor cleartext-verkeer, dwingt het gebruik van HTTPS af en voorkomt dat de app niet-versleutelde HTTP-verbindingen krijgt die worden geleverd door backend-servers.

De dart:io bibliotheek van Flutter ondersteunt HTTPS-verbinding met TLS-certificaten en de HttpClient-klasse kan worden gebruikt om HTTPS-verzoeken te doen met een aangepast vertrouwd certificaat dat wordt beheerd door SecurityContext-objecten. API-calls in Flutter kunnen dus net zo veilig zijn als vanuit native frameworks.

\textbf{Lokale authenticatie}\\
Biometrische authenticatie is geen nieuwkomer op het gebied van mobiele technologie. Het vereist dat gebruikers biologische kenmerken zoals vingerafdruk, gezicht of iris verifiëren om een gebruikersstroom te laten doorgaan. Een goed ontworpen gebruikersstroom met biometrische authenticatie verbetert de bescherming van de gevoelige informatie van de gebruiker aanzienlijk. Flutter biedt de plug-in local\_auth om middelen te bieden om lokale authenticatie op het apparaat van de gebruiker uit te voeren. In vergelijking met native frameworks biedt local\_auth meer generieke biometrische authenticatiefunctionaliteiten die voldoen aan de basisbehoeften. Om een volledig aangepaste gebruikerservaring voor biometrische authenticatie te bouwen, is het native framework nog de beste keuze.
\newpage


\section{Conclusie}
Hoe complex de app ook is en welk ontwikkelingsframework ook gekozen werd, beveiliging is de grootste zorg voor elke ontwikkelaar. Het is dus de moeite waard om wat tijd te besteden aan het lezen van veelvoorkomende maar belangrijke beveiligingsrisico's van apps en hoe deze kunnen aangepakt worden. Aangezien het gebruik van apps zal blijven toenemen, zal dit ook gelden voor de gevaren. De ontwikkelaar is dan verantwoordelijk om de gebruiker gerust te stellen door de beveiliging van de app te verhogen en ervoor te zorgen dat informatie veilig blijft. Het niet slagen ​​van de beveiligingstesten zal leiden tot gebruikers die overstappen naar de concurrentie.

Hoewel de beveiligingsopties van de native platformen talrijk zijn, moeten ontwikkelaars dezelfde mate van beveiliging kunnen bereiken met een combinatie van Flutter's framework, plug-ins en channeling naar native API's. Op dit moment vereist het ontwikkelen van Flutter apps nog steeds afzonderlijke codebases voor Android en iOS. Het combineren van rijke Flutter UI-widgets met krachtige native frameworks heeft echter een voordeel: het stelt ontwikkelaars in staat om krachtige native applicaties voor beide platforms te bouwen zonder onder te doen aan de applicatiebeveiliging. In combinatie met native frameworks heeft Flutter het potentieel en de mogelijkheid om consumenten apps van hoge kwaliteit te leveren. Google heeft het Flutter-framework gebouwd met alle beveiligingsproblemen en -fouten in gedachten. Flutter heeft bijna al de antwoorden op de meest voorkomende beveiligingsuitdagingen voor moderne apps.