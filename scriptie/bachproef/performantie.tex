%%=============================================================================
%% Performantie
%%=============================================================================


\chapter{\IfLanguageName{dutch}{Performantie}{Performance}}
\label{ch:performantie}
Dit hoofdstuk zal het performantie aspect van de applicatie behandelen. Ondanks het feit dat nieuwe smartphones alsmaar performanter worden, is het nog steeds belangrijk om belang te hechten aan de performantie van een applicatie tijdens de ontwikkeling. Hetgeen evenredig stijgt met de performantie van nieuwe smartphones, zijn de gebruikersverwachtingen van de apps. Het is belangrijk dat app ontwikkelaars de applicatie zo schrijven dat deze de beste performantie biedt aan de gebruiker. Elke app ontwikkelaar heeft namelijk als hoofddoel het ontwikkelen van een app die veel gebruikt wordt en een goede ervaring bezorgd aan de gebruiker. Wanneer een app niet voldoet aan de gebruikerseisen zal deze stoppen met de app te gebruiken, deze verwijderen en misschien zelf een negatieve review achterlaten. Wat op zijn beurt leidt tot minder gebruikers en minder downloads. 

Er zijn verschillende aspecten die de performantie van een app beïnvloeden. Deze kunnen onderverdeeld worden in verschillende categorieën. In dit onderzoek werd alleen gefocust op de app performantie. Hieronder vallen laadsnelheid, batterij verbruik, geheugen gebruik, grootte van het uitvoeringsbestand, geheugen gebruik van de app in achtergrond… Dit onderzoek spitst zich toe op opstartsnelheid, grootte van de uitvoeringsbestanden en CPU gebruik. Ook moet rekening gehouden worden met de samenhang tussen performantie en gebruikte hardware. De gebruikte hardware voor dit onderzoek werd vermeld in vorige hoofdstuk.

Het is dus belangrijk om een performante app te ontwikkelen maar het gebruikte framework zit hier ook voor een deel tussen. Het doel van dit hoofdstuk is het vergelijken van deze onderliggende verschillen. Deze worden hieronder kort toegelicht.

\section{De grootte van de uitvoeringsbestanden}
\label{sec:uitvoeringsbestanden}
Het eerste performantie aspect dat onderzocht werd was de grootte van het uitvoeringsbestand. Het doel bij applicatieontwikkeling is om een uitvoeringsbestand zo klein mogelijk te houden, zodat de gebruiker geheugen kan besparen op zijn toestel. Het is immers zo dat de prijs van high-end smartphones vandaag de dag vrij steil is. De consument kan bij het aanschaffen van een nieuw toestel vaak kiezen voor een bepaalde hoeveelheid opslag. Wanneer voor een toestel gekozen wordt met een vrij beperkte opslagcapaciteit en de gebruiker een aantal grote apps installeert, zal de opslagcapaciteit van het toestel snel volledig ingenomen zijn. Dit kan tot gevolg hebben dat de gebruiker bepaalde grotere applicaties zal gaan verwijderen om ruimte te creëren. Dit is één van de redenen waarom om een app zo klein mogelijk te houden en ondertussen de hoogste performantie te garanderen. 

Een andere reden is de maximale toegestane grootte van de uitvoeringsbestanden opgelegd door de verschillende app stores. Om een applicatie op de Google Play Store te kunnen beschikbaar stellen, moet deze voldoen aan meerdere eisen. Een van deze eisen is dat het uitvoeringsbestand van de app maximaal 100MB bedraagt. Het doel van een applicatie beschikbaar te stellen op de Play Store is immers om een zo groot mogelijk doelpubliek aan te spreken. Wanneer echter niet aan de eisen van de Play Store voldaan wordt, zal dit doelpubliek niet bereikt worden. In dit hoofdstuk zal gekeken worden naar de grootte van de uitvoeringsbestanden van respectievelijk de Flutter applicatie alsook de Android applicatie.

\subsection{Opzet}
Een uitvoeringsbestand, ook wel gekend als een executable, van een Android app kan verschillende bestandsformaten zijn. De twee meest gebruikte formaten zijn Android Package (APK) en App Bundle. Deze bestanden worden gebruikt om een Android applicatie uit te voeren. Voor dit onderzoek werd gefocust op het APK bestandsformaat. Een APK is een map of verzameling van bestanden. Al deze bestanden samen zorgen ervoor dat een applicatie kan worden uitgevoerd. Zo werden voor dit onderzoek twee APK's gegenereerd van de Hello world applicaties. De eerste vanuit de native Android codebase en de tweede van de Flutter codebase. Voor de meest bruikbare resultaten werd gekozen om voor de applicaties een release build te maken. Elke build is een verzameling van een aantal regels die worden toegepast op de code wanneer deze compileert. De release build is een verzameling van regels die de app zo compact mogelijk maakt. Dit gebeurt door het opruimen en optimaliseren van code. De resultaten van het onderzoek werden bekomen gebruik makend van de apkanalyzer, een ingebouwde Android Studio tool die de APK ontleed en de grootte van elk van deze delen weergeeft. 

Tabel \ref{table:maatstafUitvoeringsbestand} biedt inzicht in de gebruikte maatstaf voor dit deel van het onderzoek.
\begin{table}
    \begin{center}
        \caption{Gebruikte maatstaf voor de grootte van de uitvoeringsbestanden}
        \label{table:maatstafUitvoeringsbestand}
        \begin{tabular}{ |l|c|c| }
            \hline
            Bit & / & Binair getal, 1 of 0\\
            \hline
            Byte & B & 8 Bit \\ 
            \hline
            Kilobyte & KB &  \[10^{3}\]B\\ 
            \hline
            Megabyte & MB & \[10^{6}\]B \\ 
            \hline
        \end{tabular}
    \end{center}
\end{table}

\newpage

\textbf{Android}\\
In Android wordt een release build gemaakt aan de hand van volgende lijnen code in het app build.gradle bestand \ref{lst:label}.
\begin{lstlisting} [caption={Android build.gradle (app)},  label={lst:label}]
release {
    minifyEnabled true
    shrinkResources true
    proguardFiles getDefaultProguardFile
    ('proguard-android-optimize.txt'), 'proguard-rules.pro'
}
\end{lstlisting} 

\textbf{Flutter}\\
Voor een ideale Flutter release build wordt best gebruik gemaakt van onderstaande terminal commando’s.\\
\textit{flutter clean}\\
\textit{flutter build apk --split-per-abi}\\
Het flutter clean commando zal het project kleiner maken door het verwijderen van de build en .dart-tool mappen. Het commando daarna maakt een release APK per ABI. De APK’s worden gesplitst per ABI omdat dit het uitvoeringsbestand aanzienlijk verkleint. Verdere uitleg over ABI kan terug gevonden worden in hoofdstuk \ref{ch:appendix} Appendix.

Als laatste werd een opsomming gemaakt van de grootte van Flutter uitvoeringsbestanden over tijd. Dit moet een inzicht bieden in de evolutie van Flutter.

\subsection{Resultaten}
\textbf{Android}\\
Een Android app heeft geen minimum grootte voor het uitvoeringsbestand. Echter als de app op de Play Store dient gezet te worden moet deze minimum 7KB bedragen. Wanneer een nieuw Native Kotlin Android project wordt opgezet met één activity, zal dit automatisch een Hello World applicatie zijn. Als eerste stap in het onderzoek werd gekeken naar de omvang van deze APK. Volgens apkanalyzer was deze APK 3.2 MB in omvang en 2.6 MB voor de download. Vervolgens werd gekeken naar een verkleinde versie van de codebase waaruit de testing directories verwijderd werden samen met de bijhorende dependencies. Vervolgens werd minifyEnabled als ook shrinkResources op true gezet. Dit leidt tot een APK van 1.6 MB met een downloadgrootte van 1010.1KB.
Voor een derde en finale versie van de APK werd de codebase nog kleiner gemaakt, met als gevolg dat niet meer werd gewerkt volgens de Android best practices. Tabel \ref{table:androidUitvoerinsbestanden} bevat de resultaten van dit onderzoek. De code voor deze Hello world applicatie kan terug gevonden worden op GitHub repository \autocite{DePauw2021}.

\begin{table}
    \begin{center}
        \caption{Grootte van de uitvoeringsbestanden van de Hello world app in Android \autocite{DePauw2021}}
        \label{table:androidUitvoerinsbestanden}
    \begin{tabular}{ | l | m{3cm} | m{3cm} | }
        \hline
        & \textbf{APK size} & \textbf{Download Size}\\
        \hline
        META-INF & 3 KB & 3 KB\\ 
        \hline
        res & 123,7 Kb & 117 KB\\ 
        \hline
        AndroidManifest,xml & 721 B & 721 B\\ 
        \hline
        classes.dex & 293,9 KB & 293,1 KB\\ 
        \hline
        Resources.arsc & 222,6 KB & 51,7 KB\\ 
        \hline
        Kotlin & 9,1 KB & 9 KB\\ 
        \hline
        \hline
        \textbf{Total} & \textbf{693,2 KB} & \textbf{476,1 KB} \\ 
        \hline
    \end{tabular}
\end{center}
\end{table}

\textbf{Flutter}\\
Flutter maakt gebruik van de Flutter Engine (zie hoofdstuk \ref{ch:appendix} Appendix). Deze engine maakt deel uit van de APK en is nodig voor het gebruiken van een Flutter app. Deze engine bevat het gehele framework en is daarom een aantal megabyte groot. Dit maakt de Flutter APK al direct meerdere megabytes in omvang. Voor het eerste deel van het onderzoek werd gekeken naar een standaard versie van de Hello world template. Hierbij bevatte de main.dart file 26 lijnen code. De niet gesplitste ABI versie was 15.5 MB groot terwijl de gesplitste APK 5.1 MB bedroeg voor de gewone ARM en 5.5 MB voor de ARM64. Vervolgens werd de applicatie herschreven met als doel een zo klein mogelijke APK te genereren. Hierbij was het main.dart bestand 12 lijnen code na het uitvoeren van een code format. De APK versie waarbij niet gesplitst werd per ABI was 14 MB groot. De resultaten van de APK gesplitst op ABI staan vermeld in tabel \ref{table:flutterUitvoeringsbestanden}. De code voor dit onderzoek kan terug gevonden worden op de GitHub repository \autocite{DePauw2021}.

\begin{table}
    \begin{center}
        \caption{Grootte van de uitvoeringsbestanden van de Hello world app in Flutter \autocite{DePauw2021}}
        \label{table:flutterUitvoeringsbestanden}
        \begin{tabular}{ | l | m{2.5cm} | m{2.5cm} | m{2.5cm} | m{2.5cm} | }
            \hline
            & \multicolumn{2}{|c|}{\textbf{ARM}} & \multicolumn{2}{|c|}{\textbf{ARM-64}}\\
            \hline
            & \textbf{APK Size} & \textbf{Download size} & \textbf{Apk Size} & \textbf{Download Size}\\ 
            \hline
            lib & 4,2 MB & 4,2 MB & 4,7 MB & 4,6 MB\\ 
            \hline
            assets & 183,4 KB & 183 KB & 183,4 KB & 183 KB\\ 
            \hline
            META-INF & 7,9 KB & 7,6 KB & 7,9 KB & 7,6 KB\\ 
            \hline
            res & 6,2 KB & 5,9 KB & 6,2 KB & 5,9 KB\\ 
            \hline
            AndroidManifest.xml & 1021 B & 1021 B & 1022 B & 1022 B\\ 
            \hline
            classes.dex & 121,7 KB & 121,3 KB & 121,7 KB & 121,3 KB\\ 
            \hline
            Resources.arsc & 22,6 KB & 3,8 KB & 22,6 KB & 3,8 KB\\ 
            \hline
            Kotlin & 9,7 KB & 9,7 KB & 9,7 KB & 9,7 KB\\ 
            \hline
            \hline
            \textbf{Total} &  \textbf{4,6 MB} &  \textbf{4,5 MB} &  \textbf{5 MB} &  \textbf{5 MB}\\ 
            \hline
        \end{tabular}
    \end{center}
\end{table}

Op de Flutter site \autocite{Flutter} staat een korte omschrijving van de minimale downloadgrootte gemeten van een Flutter app (geen materiële componenten, slechts een enkele Center-widget, gebouwd met flutter build apk --split-per-abi), gebundeld en gecomprimeerd als een release APK. Deze minimale downloadgrootte werd in 2018 verschillende malen verkleint en opnieuw gemeten. Tabel \ref{table:flutterUitvoeringsbestandenOverTijd} bundelt de resultaten van deze metingen.

\begin{table}
    \begin{center}
        \caption{Grootte van de uitvoeringsbestanden in Flutter over tijd}
        \label{table:flutterUitvoeringsbestandenOverTijd}
        \begin{tabular}{ |l|c| }
            \hline
            \textbf{Tijd} & \textbf{Download Size}\\
            \hline
            Maart 2018 & 4,06 MB\\ 
            \hline
            Augustus 2018 & 4,20 MB\\ 
            \hline
            Begin Oktober 2018 & 4,28 MB\\ 
            \hline
            Eind Oktober 2018 & 4,48 MB\\ 
            \hline
            November 2018 & 4,70 MB\\ 
            \hline
            December 2018 & 6,70 MB\\ 
            \hline
        \end{tabular}
    \end{center}
\end{table}
\section{De opstartsnelheid van de applicatie}
\label{sec:opstartsnelheid}
Een andere veel voorkomende reden waarom gebruikers applicaties slecht beoordelen of verwijderen is een trage applicatie. Onderzoek toont aan dat gebruikers verwachten dat een applicatie binnen maximaal drie seconden opstart. Met opstartsnelheid wordt de tijd bedoelt tussen het drukken op het app icoon en het te zien krijgen van een view. Hierbij moet rekening gehouden worden met de in te laden data. Het inladen van data heeft in veel gevallen weinig te maken met app performantie. In veel gevallen zal dit te maken hebben met de aangesproken API. Een onderzoek naar de respons tijd van een API valt echter buiten het bestek van dit onderzoek. Het verschil tussen het inladen van data en het opstarten van de app kan (meestal) gezien worden aan de hand van een laad icoon. Als de app een API aanspreekt voor data zal (in de meeste gevallen) een laad icoon getoond worden terwijl tijdens de opstart procedure de app in geheugen wordt geladen en dus nog niets getoond kan worden.

Een performante app betekent in vele gevallen een snelle app, dus kan een trage opstartsnelheid leiden tot frustratie. Aangezien elke ontwikkelaar mikt op een zo goed mogelijke user experience (UX), is het interessant om ook de opstartsnelheid van een applicatie te onderzoeken. De vraag die hier beantwoord werd was volgende 'Wat is de impact van beide ontwikkelingstechnieken op de opstartsnelheid?'.

\subsection{Opzet}
Om het verschil in opstartsnelheid tussen beide platformen te testen, werd gebruik gemaakt van de Hello world applicaties. Door de kleine omvang van de applicaties, waren de verschillen in tijd beperkt. Ook werd rekening gehouden met het feit dat deze opstart tijden variabel zijn, daarom werd het onderzoek gevoerd aan de hand van een groot aantal iteraties. Hieruit konden minimum, maximum, mediaan, gemiddelde en standaard afwijking van de opstarttijden berekend worden per applicatie, die het mogelijk maakten om beide platformen met elkaar te vergelijken. 

Hierbij dient wel vermeld te worden dat beide applicaties vanaf nul gestart werden. Het is namelijk zo dat applicaties uit drie verschillende toestanden gestart kunnen worden. De Engelse termen voor deze drie opstartprocedures zijn volgende: cold start, warm start en hot start (zie hoofdstuk \ref{ch:appendix} Appendix). In dit onderzoek werd gebruik gemaakt van de cold start procedure. Deze procedure is degene die de grootste uitdaging vormt in het kader van het minimaliseren van de opstarttijd.

In Android 4.4 (API-niveau 19) en hoger bevat logcat een regel uitvoer met de waarde 'Displayed'. Deze waarde vertegenwoordigt de hoeveelheid tijd die is verstreken tussen het starten van het proces en het voltooien van het tekenen van de bijbehorende activiteit op het scherm. Dit werd gebruikt voor het bepalen van de tijd in Android.

In Flutter werd het commando flutter run --trace-startup --profile gebruikt. De trace-uitvoer wordt opgeslagen als een JSON-bestand met de naam start\_up\_info.json onder de build directory van het Flutter-project. De uitvoer geeft de verstreken tijd weer vanaf het opstarten van de app tot deze trace events. Echter toont de uitvoer van dit commando ook de tijd nodig voor Flutter om de app vanaf nul op te starten tot de eerste pixel op het scherm wordt getekend.

\subsection{Resultaten}
Zie tabel \ref{table:opstartsnelheid} voor de resultaten van het onderzoek.
\begin{table}
    \begin{center}
        \caption{Opstartsnelheden van de Android en Flutter Hello world apps. \autocite{DePauw2021}}
        \label{table:opstartsnelheid}
        \begin{tabular}{ | l | l | l |}
            \hline
            & \textbf{Android} & \textbf{Flutter}\\
            \hline
            \textbf{Hoogst} & 985ms & 848ms\\
            \hline
            \textbf{Laagst} & 582ms & 293ms\\
            \hline
            \textbf{Mediaan} & 649ms & 416,5ms\\
            \hline
            \textbf{Gemiddeld} & 681,31ms & 454,38ms\\
            \hline
            \textbf{Standaard afwijking} & 89,6450571ms & 126,5891134ms\\
            \hline
        \end{tabular}
    \end{center}
\end{table}

\section{Discussie}
Uit onderzoek van Bracke \autocite{Bracke2020} bleek dat in 2020 de opstartsnelheid van native Android apps gemiddeld 146,7ms lager was dan Flutter apps. Maar aan de andere kant dat de Flutter apps een stabielere opstarttijd ondervonden. Dit is volledig het tegenovergestelde met wat dit onderzoek aantoont. De resultaten zijn volledig anders wat doet vermoeden dat de omgevingsvariabelen niet hetzelfde zullen geweest zijn. Een andere mogelijkheid is de onderliggende hardware die hier een te grote rol speelde.
\section{Het CPU gebruik van de applicatie}
\label{sec:cpu}
Een Central Processing Unit of kortweg CPU is het als het ware het brein van een computer. Het is een stuk hardware dat aan de hand van een aantal basisoperaties, programma’s uitvoert. Om deze operaties performant uit te voeren heeft de CPU echter nood aan een vaste, degelijke bron van energie. In het geval van mobiele apparaten, zoals een laptop of een smartphone, is deze energiebron de batterij van het toestel. Vandaag de dag moeten, en zijn, smartphones vrij tot zeer compact. Dit heeft echter ook een nadeel. Door de compactheid van de toestellen, is de plek voor de batterij ook eerder beperkt. Om de capaciteit van de batterij zo goed mogelijk te benutten, moeten de processen die op de toestellen draaien dus eerder performant zijn. Hiermee wordt bedoeld dat ze energiezuinig moeten zijn zonder snelheid in te leveren. 

Een app die veel CPU werk vereist zal dus meer energie verbruiken, de batterij sneller doen leeglopen en kan zorgen voor oververhitting van de batterij. Een probleem dat voorkomt wanneer de app meer resources gebruikt als nodig.

Aangezien het beheer van de batterijduur een dagelijkse ervaring is voor velen, wil men zo weinig mogelijk batterij verspelen aan korte app interacties. Gebruikers die merken dat een app te veel vraagt van de CPU zullen deze app dan ook sneller verwijderen. Daarom was het interessant om te onderzoeken of de onderliggende frameworks hier een aandeel in hebben. 

\subsection{Opzet}
Voor het onderzoek naar CPU gebruik werden de onderzoeksapplicaties vergeleken. Deze applicaties zijn omvangrijk en bevatten meerdere features die CPU intensieve taken uitvoeren. 

Het CPU gebruik wordt uitgedrukt in procent. Hoe hoger dit getal hoe meer CPU de app in gebruik nam. De gehele app werd doorlopen en het gemiddelde CPU gebruik werd vergeleken. Uitschieters werden ook opgenomen in de resultaten voor het maken van vergelijkingen omtrent CPU intensieve components.

Voor het onderzoek naar CPU gebruik werd voor Android gebruik gemaakt van de Android Profiler. Een krachtige monitoring tool die het mogelijk maakt om live CPU, geheugen, netwerk en batterij resources op te volgen. Meer specifiek werd gekeken naar de CPU Profiler. Voor Flutter werd gebruik gemaakt van de Flutter DevTools. Een uitgebreide tool die in Flutter 2.0 een grote update kende. Hiermee kunnen live zaken zoals logging, geheugen gebruik, CPU gebruik, netwerk requests… worden opgevolgd.

\subsection{Resultaten}
Zie tabel \ref{table:cpu}
\begin{table}
    \begin{center}
        \caption{Opstartsnelheid}
        \label{table:cpu}
        \begin{tabular}{ | l | l | l |}
            \hline
            & \textbf{Android} & \textbf{Flutter}\\
            \hline
            \textbf{Hoogst} & \% & \%\\
            \hline
            \textbf{Laagst} & \% & \%\\
            \hline
            \textbf{Mediaan} & \% & \%\\
            \hline
            \textbf{Gemiddeld} & \% & \%\\
            \hline
            \textbf{Standaard afwijking} & \% & \%\\
            \hline
        \end{tabular}
    \end{center}
\end{table}


Tabel \ref{table:cpu} laat zien dat Flutter en Android in dit criterium niet veel van elkaar verschillen. De Flutter applicatie had een lager maximaal CPU-gebruik, maar de native Android had een lagere gemiddelde waarde en beide hadden dezelfde laagste waarde. De Android-applicaties hadden over het algemeen in het begin hoge CPU-prestaties.

\section{Conclusie}
\label{sec:perfConclusie}
bij grotere Flutter APK’s wordt het verschil klein en insignificant.

Wat uit deze resultaten kan geconcludeerd worden is dat een duidelijk verschil is tussen enerzijds de native APK’s en de cross-platform APK’s. Het is namelijk zo dat native APK’s aanzienlijk kleiner zijn in vergelijking met hun cross-platform tegenhangers.