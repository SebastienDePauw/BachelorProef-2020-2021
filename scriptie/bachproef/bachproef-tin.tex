%===============================================================================
% LaTeX sjabloon voor de bachelorproef toegepaste informatica aan HOGENT
% Meer info op https://github.com/HoGentTIN/bachproef-latex-sjabloon
%===============================================================================

\documentclass{bachproef-tin}

\usepackage{hogent-thesis-titlepage} % Titelpagina conform aan HOGENT huisstijl

%%---------- Documenteigenschappen ---------------------------------------------

% De titel van het rapport/bachelorproef
\title{Titel}

% Je eigen naam
\author{Sébastien De Pauw}

% De naam van je promotor (lector van de opleiding)
\promotor{Ozgür Akin}

% De naam van je co-promotor. Als je promotor ook je opdrachtgever is en je
% dus ook inhoudelijk begeleidt (en enkel dan!), mag je dit leeg laten.
\copromotor{Navaron Bracke}

% Indien je bachelorproef in opdracht van/in samenwerking met een bedrijf of
% externe organisatie geschreven is, geef je hier de naam. Zoniet laat je dit
% zoals het is.
\instelling{Next Apps BV}

% Academiejaar
\academiejaar{2020-2021}

% Examenperiode
%  - 1e semester = 1e examenperiode => 1
%  - 2e semester = 2e examenperiode => 2
%  - tweede zit  = 3e examenperiode => 3
\examenperiode{2}

%===============================================================================
% Inhoud document
%===============================================================================

\begin{document}

%---------- Taalselectie -------------------------------------------------------
% Als je je bachelorproef in het Engels schrijft, haal dan onderstaande regel
% uit commentaar. Let op: de tekst op de voorkaft blijft in het Nederlands, en
% dat is ook de bedoeling!

%\selectlanguage{english}

%---------- Titelblad ----------------------------------------------------------\cite{Flutter2021}
\inserttitlepage

%---------- Samenvatting, voorwoord --------------------------------------------
\usechapterimagefalse
%%=============================================================================
%% Voorwoord
%%=============================================================================

\chapter*{\IfLanguageName{dutch}{Woord vooraf}{Preface}}
\label{ch:voorwoord}

%% TODO:
%% Het voorwoord is het enige deel van de bachelorproef waar je vanuit je
%% eigen standpunt (``ik-vorm'') mag schrijven. Je kan hier bv. motiveren
%% waarom jij het onderwerp wil bespreken.
%% Vergeet ook niet te bedanken wie je geholpen/gesteund/... heeft

Deze bachelorproef werd voltooid in kader van de opleiding Toegepaste Informatica met specialisatie mobiele applicaties. Mijn nieuwsgierigheid naar cross-platform development samen met mijn passie voor Android leidde tot de keuze voor dit onderwerp. Ik geloof dat cross-platform development de toekomst is. De kostbesparende optie waar elke klant naar zoekt. Dit onderzoek moest laten uit blijken of Flutter al ver genoeg ontwikkeld is om te gebruiken in professionele omgeving. Dit onderzoek heeft mij veel bijgeleerd over de werking van beide platformen alsook verschillende programmeerconcepten. Dit bracht mij een stap dichter bij de volwaardige programmeur die ik na deze opleiding wens te worden.

Graag zou ik van dit voorwoord ook nog gebruik willen maken om een aantal mensen te bedanken. Eerst en vooral zou ik Navaron Bracke willen bedanken. Zijn kennis, hulp en feedback werden enorm geapprecieerd en hielpen mij een andere kijk werpen op het onderzoek. Vervolgens zou ik mevrouw Akin willen bedanken voor de tijd en moeite die zij stak in het lezen en verbeteren van deze bachelorproef. Ook wil ik Next Apps bedanken voor het voorleggen van het onderwerp. De kennis die zij mij bijbrachten over Android was van grote hulp tijdens dit onderzoek. Daarnaast wil ik Kilian Hoefman persoonlijk bedanken voor de hulp bij dit onderzoek maar ook voor de steun doorheen de jaren. Tot slot wil ik mijn vrienden en familie bedanken voor alle hulp en steun doorheen deze opleiding. Deze mensen hielpen deze Bachelorproef vormen tot wat hij nu is.
%%=============================================================================
%% Samenvatting
%%=============================================================================

% TODO: De "abstract" of samenvatting is een kernachtige (~ 1 blz. voor een
% thesis) synthese van het document.
%
% Deze aspecten moeten zeker aan bod komen:
% - Context: waarom is dit werk belangrijk?
% - Nood: waarom moest dit onderzocht worden?
% - Taak: wat heb je precies gedaan?
% - Object: wat staat in dit document geschreven?
% - Resultaat: wat was het resultaat?
% - Conclusie: wat is/zijn de belangrijkste conclusie(s)?
% - Perspectief: blijven er nog vragen open die in de toekomst nog kunnen
%    onderzocht worden? Wat is een mogelijk vervolg voor jouw onderzoek?
%
% LET OP! Een samenvatting is GEEN voorwoord!

%%---------- Samenvatting -----------------------------------------------------
% De samenvatting in de hoofdtaal van het document

\chapter*{\IfLanguageName{dutch}{Samenvatting}{Abstract}}
Vier jaar geleden werd Flutter uitgebracht door Google. In deze vier jaar werd Flutter de meest gebruikte cross-platform ontwikkelings-stack. Het overtreft zijn concurrentie, sommige die al meer als 10 jaar bestaan. 

% - Context: waarom is dit werk belangrijk?


% - Nood: waarom moest dit onderzocht worden?

% - Taak: wat heb je precies gedaan?

% - Object: wat staat in dit document geschreven?

% - Resultaat: wat was het resultaat?

% - Conclusie: wat is/zijn de belangrijkste conclusie(s)?

% - Perspectief: blijven er nog vragen open die in de toekomst nog kunnen
%    onderzocht worden? Wat is een mogelijk vervolg voor jouw onderzoek?
Echter zijn na dit onderzoek niet alle vragen beantwoord. 

%---------- Inhoudstafel -------------------------------------------------------
\pagestyle{empty} % Geen hoofding
\tableofcontents  % Voeg de inhoudstafel toe
\cleardoublepage  % Zorg dat volgende hoofstuk op een oneven pagina begint
\pagestyle{fancy} % Zet hoofding opnieuw aan

%---------- Lijst figuren, afkortingen, ... ------------------------------------

% Indien gewenst kan je hier een lijst van figuren/tabellen opgeven. Geef in
% dat geval je figuren/tabellen altijd een korte beschrijving:
%
%  \caption[korte beschrijving]{uitgebreide beschrijving}
%
% De korte beschrijving wordt gebruikt voor deze lijst, de uitgebreide staat bij
% de figuur of tabel zelf.

\listoffigures
\listoftables

% Als je een lijst van afkortingen of termen wil toevoegen, dan hoort die
% hier thuis. Gebruik bijvoorbeeld de ``glossaries'' package.
% https://www.overleaf.com/learn/latex/Glossaries

%---------- Kern ---------------------------------------------------------------

%%=============================================================================
%% Inleiding
%%=============================================================================

\chapter{\IfLanguageName{dutch}{Inleiding}{Introduction}}
\label{ch:inleiding}

De impact van mobiele applicaties (vanaf nu apps) op ons alledaags leven is niet te onderschatten. Ze zijn de dag van vandaag niet meer weg te denken. Dit komt door de grootte toename in de verkoop van smartphones. Ongeveer 40\% van de wereldbevolking beschikt over een smartphone en in 2020 gingen er 1.38 miljard nieuwe toestellen de deuren uit. De toename in verkochte smartphones leidt tot een grotere markt voor apps. Dit hebben softwarebedrijven niet over het hoofd gezien. Het aantal applicaties aanwezig op de iOS App Store en de Google Play Store, is in de voorbije tien jaar respectievelijk vertwintig- en verdertigvoudigd. Elke smartphone maakt gebruik van een besturingssysteem, dit is de software die de hardware aanstuurt. Tegenwoordig is er een oligopolie van twee spelers op de markt van smartphone besturingssystemen (vanaf nu OS voor Operating System). Apple ontwikkelde hun eigen OS genaamd iOS terwijl de meeste andere merken zoals Samsung en OnePlus gebruik maken van het Android OS. Elk OS heeft een verschillende manier van applicatieontwikkeling. In de meeste gevallen zullen app eigenaars met hun applicatie een zo breed mogelijk publiek willen bereiken. Hierdoor zullen ze apps moeten ontwikkelen voor zowel iOS en Android.

\newpage
\textbf{Wat is Native Development?}\\
Als het ontwikkelingsteam kiest om een applicatie native te gaan ontwikkelen dan wil dit zeggen dat een aparte codebase (verklaren) moet ontwikkeld worden voor elk mobiel besturingssysteem waar de app moet op kunnen draaien. Verschillende codebases wil zeggen meer ontwikkelings- en onderhoudswerk maar geeft anderzijds wel een optie tot een native- design en gedrag per codebase.

\textbf{Wat is Cross-platform Development?}\\
Een andere aanpak is het ontwikkelen van een cross-platform app. Dit is het schrijven van één codebase die kan gecompileerd (vertaald) worden naar verschillende besturingssystemen. Het is een optie die de laatste jaren meer en meer gekozen wordt. De voor- en nadelen die hiervoor zijn aangehaald, draaien zich om. Een enkele codebase is goedkoper om te bouwen en te onderhouden maar dit moet dan ook afgewogen worden tegen het laten van native- designs en gedrag. Er zijn verschillende ontwikkeltools gemaakt voor het maken van cross-platform apps. Een paar van de grootste zijn React Native, Xamarin, Flutter en Ionic.

\textbf{Wat is Flutter}\\
Flutter is een Google UI toolkit voor het bouwen van mooie, bijna natively gecompileerde apps voor mobiel, web, en desktop en dit alles van één enkele codebase. Flutter kondigde op 3 maart 2021 hun nieuwe update aan die hen weer een stap in de goede richting duwde. Dit is onder andere te danken aan de grootte steun en de snelle tractie van het platform in de voorbije jaren. Flutter is opensource wat wil zeggen dat ze hun broncode openstellen voor het publiek, waardoor iedereen het kan bekijken. Zo kunnen verbeteringen worden gedaan door eindgebruikers die het beste voor hebben met het platform. Als Flutter deze verbeteringen goedkeurt worden deze opgenomen in de broncode en beschikbaar gesteld voor iedereen.

\textbf{Waarom Flutter versus native Android?}\\
Flutter en Android zijn beiden ontwikkeld door Google. Alhoewel Flutter relatief jong is, kan dit niet gezegd worden van native Android. Het platform kende al veel updates en staat al lang niet meer in zijn jonge schoentjes. Vandaar dat het toetsen van deze twee een interessante kijk kan geven op het jonge flutter platform. De Flutter applicatie zal geschreven worden voor- en alleen gecompileerd worden naar Android. De sterkte van Cross-platform development is natuurlijk één codebase voor verschillende besturingssystemen, maar voor dit onderzoek is alleen Android van belang. Het onderzoek zal rekening houden met het geschreven aantal lijnen code per uitgewerkt hoofdstuk. Als het onderzoek rekening moet houden met iOS brengt dat het evenwicht uit balans.

\textbf{Waarom Kotlin Android en geen Java Android?}\\
Deze keuze is voor velen persoonlijk alhoewel hier ook onderzoek naar verricht is. In deze studie wordt Kotlin gebruikt omdat de syntax van de taal zal helpen bij het intomen van het aantal lijnen code. Hoe minder lijnen code, hoe eenvoudiger en overzichtelijker de app. Het is voor velen dan ook de voorkeurstaal om Android-applicaties in te ontwikkelen. Dit onderzoek zal geen stap voor stap uitleg inhouden van hoe de code in elkaar zit. Het doel is een vergelijking schetsen tussen Flutter en Android. De code die wordt geschreven zal dus alleen worden aangehaald indien dit relevant is voor de studie.


\section{\IfLanguageName{dutch}{Probleemstelling}{Problem Statement}}
\label{sec:probleemstelling}

Het te onderzoeken domein kwam vanuit Next Apps. Een native app ontwikkelingsbedrijf uit Lokeren dat zich specialiseert in Android en iOS watch, phone en tablet apps. Zij merken dat niet alle klanten op de hoogte zijn van het hoge kostenplaatje van een native ontwikkelde app. Om in de toekomst een middenweg te vinden en de kost van een kleine app te dempen, denken zij erover om sommige apps cross-platform te gaan ontwikkelen. Bij het onderzoeken van de cross-platform markt kwamen ze Flutter tegen, het nieuwe platform met veel potentieel. Daarom werd de vraag gesteld om de stand van het Flutter platform te onderzoeken. Met deze scriptie hopen ze een duidelijk beeld te krijgen op de limieten en mogelijkheden van het framework.

\section{\IfLanguageName{dutch}{Onderzoeksvraag}{Research question}}
\label{sec:onderzoeksvraag}

\subsection{Hoofdonderzoeksvraag}

\textbf{Wat zijn de voor- en nadelen van app ontwikkeling in Flutter in vergelijking met native Android?}\\
Het onderzoek zal zich vooral richten op het vinden van een antwoord op deze hoofdvraag. Aan de hand van deze kan een duidelijk beeld worden geschetst van de huidige stand van Flutter in vergelijking met Android. Hierbij wordt gelet op de ontwikkelingstijd van de features, de complexiteit van de code, het aantal lijnen geschreven lijnen code…
Hypothese: de voorkeur van ontwikkelingsplatform zal hoogstwaarschijnlijk afhangen van de soort app die ontwikkeld moet worden. Een grootere app met veel features zal eerder native ontwikkeld worden, terwijl kleinere apps die weinig native behaviour kunnen bevatten waarschijnlijk cross-platform ontwikkeld zullen worden.

\subsection{Deelonderzoeksvragen}

\textbf{Is Flutter al matuur genoeg om te aanschouwen als volwaardig alternatief op native app ontwikkeling?}\\
Deze vraag biedt een antwoord aan alle native app ontwikkelaars die denken om de overstap te maken naar cross-platform development. Het is belangrijk om een overzicht te krijgen in de limieten van een platform alvorens er tijd en geld in te investeren. 
Hypothese: Flutter zal geen duidelijke standaard hebben. Best practices, goed ondersteunde libraries en coding principles zijn nog niet gedefinieerd door het platform waardoor het voor beginners onduidelijk zal lijken wat de beste manier van aanpak zal zijn. Ondanks de snelle groei en de grootte steun van het platform wordt wel verwacht dat hier snel verandering in gebracht wordt.

\textbf{Is Flutter toegankelijk voor nieuwe ontwikkelaars?}\\
Het proces van het ontwikkelen van de Flutter applicatie zal antwoord bieden op deze vraag. Voor velen is het moeilijk om te schakelen naar een nieuw platform met een nieuwe taal. Om deze overgang transparant te maken, wordt een beeld geschetst van de moeilijkheidsgraad van Flutter en Dart. Hierdoor kan de lezer zelf beslissen of hij deze keuze wil maken.
Hypothese: voor ontwikkelaars met een Java of C++ achtergrond zou de Dart taal geen probleem mogen vormen. Ze hebben een gelijkaardige syntax. Verwachtingen van het Flutter platform liggen hoog. De snelle groei van de gebruikersbasis doet vermoeden dat het een toegankelijk platform is.


\section{\IfLanguageName{dutch}{Onderzoeksdoelstelling}{Research objective}}
\label{sec:onderzoeksdoelstelling}

Zoals hiervoor vermeld zal dit onderzoek de voor- en nadelen van het Flutter framework proberen vinden. Om dit op een duidelijke manier te doen, zal het framework getoetst worden aan native Android. Door het jonge platform af te wegen tegen het ontwikkelde native tegengestelde, wordt gehoopt op een duidelijk beeld van Flutter. Het zal interessant zijn om te weten of het platform te kort doet aan native Android, want deze dure tegenhanger moet zo goed mogelijk zijn troeven uitspelen om niet ten onder te gaan aan het goedkope alternatief. Het onderzoek wordt gevoerd aan de hand van een vergelijkende studie. 

Voor het onderzoek worden twee identieke applicaties ontwikkeld. Beide applicaties zullen ontwikkeld worden in Android studio maar de ene zal geschreven worden in Kotlin terwijl de andere geschreven wordt in Dart. Door het ontwikkelproces van de Flutter app af te toetsen tegen dat van de native Android app zal hopelijk een beeld worden geschetst van de stand van het Flutter framework. Met dit beeld zal dan een antwoord gevormd worden op de onderzoeksvraag.

Dit onderzoek gaat gepaard met een deadline, deze tijdsdruk zorgt ervoor dat een aantal criteria moeten gesteld worden alvorens te starten. Deze criteria zullen zorgen voor een kwaliteitsvol eindresultaat dat haalbaar is in het opgelegd tijdsbestek. Het eerste aantal criterium gaat over de performantie van de twee apps. Dit criterium wordt verder onderverdeeld in drie verschillende criteria. Allereerst wordt gekeken naar de opstartsnelheid van de apps. Vervolgens wordt de grootte van de uitvoeringsbestanden bekeken. Als laatste wordt gekeken naar het CPU gebruik van beide apps. De volgende criteria is creatie van views, hier wordt gekeken hoe een layout zich vertaalt naar een view. Hoe complexe views gemaakt worden en wat de mogelijkheden en grenzen zijn van cross-platform layouts. 

Het beoogde resultaat van de studie is een antwoord bieden op alle onderzoeksvragen. Indien dit lukt, kan de studie als succesvol worden beschouwd, anders zal een suggestie worden gedaan voor toekomstig bijkomend onderzoek.

\newpage
\section{\IfLanguageName{dutch}{Opzet van deze bachelorproef}{Structure of this bachelor thesis}}
\label{sec:opzet-bachelorproef}

% Het is gebruikelijk aan het einde van de inleiding een overzicht te
% geven van de opbouw van de rest van de tekst. Deze sectie bevat al een aanzet
% die je kan aanvullen/aanpassen in functie van je eigen tekst.

Deze scriptie kan opgedeeld worden in verschillende hoofdstukken, elk hoofdstuk kan bestaan uit verschillende secties of ondertitels. Deze sectie beschrijft de inhoud van elk van deze hoofdstukken.

Hoofdstuk~\ref{ch:inleiding}: Inleiding: schetst een kort beeld van de inhoud van deze studie. Biedt een antwoord op sommige vragen rond deze paper.

Hoofdstuk~\ref{ch:stand-van-zaken} Stand van zaken: bevat een literatuurstudie. Een samenvatting van relevante onderzoeken die reeds voorafgingen aan dit onderzoek.

Hoofdstuk~\ref{ch:methodologie} Methodologie: omschrijft hoe het onderzoek in zijn werk zal gaan. De opzetting van het onderzoek en het onderzoek zelf zullen duidelijk toegelicht worden.

Hoofdstuk 4 Grootte van de uitvoeringsbestanden en opstartsnelheid van de app: kijkt naar twee performantie criteria van de app. Grootte van de uitvoeringsbestanden zal de omvang van twee identieke app APK’s vergelijken.

Hoofdstuk 5 CPU gebruik van de app:

Hoofdstuk 6 Gebruik van online API’s:

Hoofdstuk 7 Security:

Hoofdstuk 8 Creatie van views:

Hoofdstuk 9 Beschikbare libraries en code complexiteit:

Hoofdstuk 10 Asynchroon werken:

Hoofdstuk 11 Beschikbare tools:

Hoofdstuk 12 Appendix: zal verdere uitleg bieden rond bepaalde onderwerpen voor de geïnteresseerde lezer.

Hoofdstuk~\ref{ch:conclusie} Conclusie: dit is de algemene conclusie op deze scriptie. In deze conclusie zal een antwoordt gegeven worden op de reeds aangehaalde onderzoeksvragen. Daarbij wordt ook aangezet tot toekomstig onderzoek binnen dit vakgebied.
%%=============================================================================
%% Stand van zaken
%%=============================================================================


\chapter{\IfLanguageName{dutch}{Stand van zaken}{State of the art}}
\label{ch:stand-van-zaken}

% Tip: Begin elk hoofdstuk met een paragraaf inleiding die beschrijft hoe
% dit hoofdstuk past binnen het geheel van de bachelorproef. Geef in het
% bijzonder aan wat de link is met het vorige en volgende hoofdstuk.

% Pas na deze inleidende paragraaf komt de eerste sectiehoofding.

Dit hoofdstuk is equivalent aan een literatuurstudie. De eerste stap in dit onderzoek was het zoeken naar voorgaande studies over Flutter en Android. Elke studie die voldeed aan de eisen en als interessant werd beschouwd, werd opgeslagen en gebundeld. Hiervan werd de inhoud diagonaal gelezen om zo alleen de meest relevante onderzoeken over te houden.

Vervolgens werd een opsomming gemaakt en een kort beeld geschetst van een aantal van deze studies en hun bijdrage in het vakgebied. In de volgende sectie zullen dus alleen de meest relevante studies, die dicht aansluiten bij dit onderzoek, worden aangehaald. De ondervindingen uit deze studies werden gebruikt als voorbereiding op het onderzoek en kunnen in volgende hoofdstukken gebruikt worden als steunpunten. Door de ondervindingen van verschillende studies te bundelen, kon gezocht worden naar een rode draad.

De te onderzoeken criteria van deze studie werden vastgelegd op basis van de reeds voorgaande onderzoeken. Het is de bedoeling dat deze studie bijdraagt aan het vakgebied. Hierdoor wordt weerhouden van reeds onderzochte criteria opnieuw te gaan onderzoeken. Toch zal blijken dat sommige onderzochte criteria opnieuw worden bekeken. Deze herhaling komt voort uit de Flutter 2.0 release, deze zorgde voor veranderingen in de prestatiemogelijkheden van het framework.\newpage

\section{Voorgaande onderzoeken}
\label{sec:voorgaande-onderzoeken}
 Navaron Bracke onderzocht de verschillen en gelijkenissen tussen Android en Flutter in een vergelijkende studie. \autocite{Bracke2020}. De scriptie vormde een beeld over de toenmalige stand van het Flutter framework door het te toetsen tegen het ontwikkelde Android. Het onderzoek was afgebakend door een vooraf vastgelegd aantal criteria. Zo kon de onderzoeker zekerheid bieden over de grondigheid van de studie. De onderzochte criteria waren: Internationalisering, navigatie, persisteren van gegevens, user Interfaces, asynchroon werk, permissies, software testing, opstarttijd van de applicatie, grootte van de applicatie. De opzet en het doel van het onderzoek liep in grootte mate gelijk met het onderzoek gevoerd voor deze scriptie. Daarom was het onderzoek van Bracke uitermate interessant, bevindingen werden bijgehouden, genoteerd en getoetst tegen de bevindingen van dit onderzoek. De hoofdstukken die performantie onderzochten werden gebruikt om een beeld te schetsen van de Flutter evolutie over tijd. De andere twee hoofdstukken (User Interfaces en Asynchroon werk) werden opnieuw onderzocht met conclusies van Bracke in het achterhoofd. Hiervoor werd, in de mate van het mogelijke, zoveel mogelijk gefocust op de onderwerpen die Bracke niet aanhaalde. De onderzoeker constateerde dat, ondanks de jonge aard van het Flutter platform, het een goed alternatief biedt op elk onderzocht criterium. Bij elk van deze criteria werd een platform van voorkeur gekozen. Hieruit bleek Flutter op 4 van de 7 criteria naar voor te komen. Al werd wel meegedeeld dat niet alle features in Flutter even gebruiksvriendelijk zijn maar door de jonge aard werd verwacht dat dit in de toekomst zou verbeteren.

Daniel Dang en David Skelton, twee professors aan de Insitutue of Technology, schreven een wetenschappelijk artikel over mobile app development in het onderwijs \autocite{Dang2019}. Het doel van het onderzoek was het bepalen van het ideale framework voor het aanleren van mobiele applicatieontwikkeling in het onderwijs. Het onderzoek werd gevoerd aan de hand van verschillende native- en cross-platform frameworks. Zo werden native Android, native iOS, Flutter, React Native, Ionic, Xamarin, Cordova en Appcelerator vergeleken. Onderzochte criteria bij dit onderzoek waren: Design App User Interface, User Event Handling, Services \& Multiple threads used in Internet connection, Graphics and Animation, Device hardware and sensor, Wireless connectivity, Internal data storage, Real time database. Dit artikel werd gebruikt als basis voor het sommige delen van het onderzoek gevoerd in deze scriptie. Zo werden de hoofdstukken User Event Handling en Services \& Multiple threads used in Internet connection grondig gelezen en ontleed voor dit onderzoek. In conclusie bleken Flutter en Native Android de twee beste platformen te zijn voor gebruik tijdens praktijklessen. Ook beval het onderzoek acht ideale onderwerpen aan die in de praktijksessies moeten worden behandeld om studenten voldoende technische vaardigheden bij te brengen om alle soorten mobiele apps te ontwikkelen. Belangrijk hierbij te vermelden is dat de studie gebruik maakte van een oude Flutter release.

Austen Latture onderzocht de ontwikkelcyclus van een app in Flutter \autocite{Latture2020}. De focus van het onderzoek lag hem op de leercurve van het platform. Hier waren geen vooropgestelde criteria aanwezig, deze studie omschreef de uitwerking van de Backdrop app. In conclusie bleek Flutter een toegankelijk platform te zijn voor nieuwe ontwikkelaars met een allesbehalve steile leercurve. Een ander groot voordeel aangeboden door het platform is de talrijke ingebouwde functionaliteit. De Dart taal die wordt geschreven voor het maken van Flutter apps is volgens het onderzoek gemakkelijk aan te leren voor mensen met een achtergrond in de C taal. Aan de andere kant bleek het kiezen van third-party libraries een hinder blok te vormen tijdens de ontwikkelfase. De jonge aard van het platform ging gepaard met jonge libraries. Zoals bleek uit andere studies concludeerde dit onderzoek dat Flutter hier een rode draad in mist. De grootte hoeveelheid aangeboden libraries maakt het moeilijk om te weten welke goed ondersteund worden en dan ook vaak gekozen worden. Dit is vooral een struikelblok voor nieuwe developers, onbekend met het platform. Vervolgens werd Flutter ook in het kort vergeleken met React Native. Hieruit kon geconcludeerd worden dat Flutter veel gemakkelijker te installeren is. Dit komt, volgens het onderzoek, voort uit de ingebouwde functionaliteit van Flutter terwijl React meer werkt met verspreide derde partij software die meestal bestaat uit te veel boilerplate code. Het onderzoek van Latture hielp dit onderzoek met antwoorden op één van de deelonderzoeksvragen 'Is Flutter toegankelijk voor nieuwe ontwikkelaars?'. Ondanks de subjectiviteit van de paper, werd wel rekening gehouden met de bevindingen van het onderzoek. Echter kon niet naast de onwetenschappelijke aard van het onderzoek gekeken worden, hierdoor wordt verder niet gerefereerd naar deze bevindingen.

Carlos Chavez, een student en Yoonsik Cheon, een professor aan de Universiteit van Texas El Paso onderzochten het herschrijven van Native Android apps naar Flutter apps \autocite{Cheon2020}. Het onderzoek bekeek een native Android app ontwikkeld in de Java taal en probeerde deze zo goed mogelijk om te zetten naar een Flutter app die terug kon gecompileerd worden als een Android app maar ook als iOS app. Het onderzoek werd gevoerd aan de hand van het Flutter ontwikkelproces. De technische-, praktische problemen en uitdagingen die de kop opstaken werden vervolgens besproken. Het onderzoek deed een paar duidelijke verschillen uit blijken tussen Android en Flutter wanneer gekeken werd naar de taal, programmeerstijl en design. Het grootste werk was het herschrijven van de views terwijl het herschrijven van de achterliggende logica eerder snel gebeurd was. Het onderzoek focuste zich onder andere op het aantal lijnen code geschreven voor beide applicaties. De Flutter code bleek hierbij maar 2/3 van het totaal aantal Android code te zijn. Maar volgens de onderzoekers was dit niet genoeg om apps in Flutter te beginnen ontwikkelen. De leeftijd van het platform bracht nog te veel onzekerheid met zich mee. Ook ontbrak dit onderzoek aan een soort standaard werkwijze of verzameling van richtlijnen. Dit sloeg dan vooral op libraries en API’s binnen het platform. De gebruikersbasis en de beschikbare tools moesten hierbij ook zeker in rekening worden gebracht. Het grote aantal third party libraries, door de grootte aanhang, zorgde voor onzekerheid bij het zoeken naar een goed ondersteunde library. Bij andere grootte platformen wordt dit verholpen doordat na bepaalde duur de beste API’s en libraries naar boven komen en een soort van community standard wordt gezet. Dit onderzoek was uitermate interessant en legde gronde voor een groot deel van dit onderzoek. Ook al zijn geen raakvlakken aanwezig tussen de twee onderzoeken, veel kennis vergaard uit deze paper werd gebruikt als bouwsteen van dit onderzoek.

Mathilda Olsson onderzocht hoe Flutter applicaties vergelijken met native applicaties \autocite{Olsson2020}. Opnieuw een onderzoek dat dicht aansluit bij het onderzoek gevoerd voor deze scriptie. Voor haar onderzoek ontwikkelde Olsson twee native applicaties, één in Kotlin Android en één in Swift iOS. Deze applicaties werden vervolgens beoordeelt op vlak van CPU performantie. Uit onderzoek bleek dat Flutter en Android apps gelijkaardige resultaten halen op vlak van CPU performantie. Vervolgens werd gekeken naar het aantal lijnen code en de complexiteit van de twee code bases. Hieruit bleek dat Flutter aanzienlijk minder lijnen code nodig had maar ook minder complexe code bevatte. De Flutter app bestond uit 125 lijnen code terwijl de twee native apps samen 580 lijnen code bevatte. Dit verschil is substantieel en duidt erop dat Flutter beter blijkt voor kleine tot middelgrote applicaties. De bevindingen over CPU performantie werden gebruikt in dit onderzoek voor het schetsen van een beeld van CPU performantie over tijd. Daarbuiten bleek Flutter opnieuw een betere optie over Android, wat leidde tot het vormen van een hypothese op de onderzoeksvraag. Het onderzoek van Olsson bevatte een bevraging die zich focuste op de verschillen in gebruikersperceptie. Een aantal gebruikers moesten de app eerst testen waarna ze een vragenlijst konden beantwoorden waaruit verschillen in look en feel moesten duidelijk worden. Uit de vragenlijst bleek native ontwikkeling, op vlak van UI, de voorkeur te hebben van het grootste deel gebruikers. De verschillen waren vooral te zien bij animaties, fonts, gedrag, lijsten en menu’s. Animaties konden wel gelijkgesteld worden maar dat is extra werk zijn voor de ontwikkelaar. In conclusie, zoals in de andere onderzoeken, werd nogmaals geduid op de leeftijd van het platform. Alle criteria waar Flutter minder op scoorde konden nog volop in ontwikkeling zitten voor een volgende release. Flutter had volgens het onderzoek veel potentieel indien de steun van de gebruikersbasis groot bleef.

Jakhongir Fayzullaev onderzocht drie verschillende cross-platform development frameworks, Kotlin Native, Multi-OS en Flutter in een vergelijkende studie \autocite{Fayzullaev2018}. De frameworks en dus de studie is vooral interessant voor Android ontwikkelaars aangezien de onderzochte frameworks dicht aansluiten bij de Java en Android ontwikkeling methoden. In 2018, toen het onderzoek werd uitgevoerd, waren de Flutter en Multi-OS nieuwe spelers op de cross-platform markt. Het onderzoek kon zich niet baseren op voorgaande studies en ontbrak aan duidelijke richtlijnen van de frameworks. Hierdoor was het onderzoek oppervlakkig, het vergeleek de grote lijnen van de platformen zonder de details te bekijken. Het Flutter deel van het onderzoek bekeek in grote lijnen: Widgets, het maken van layouts en de bouwstenen van Flutter. Fayzullaev concludeerde dat op vlak van performantie de drie platformen zo goed als identiek waren. Onderzoek deed blijken dat Flutter een goed platform was in 2018 maar de jonge aard zorgde voor weinig bruikbare libraries en tools. Ook had Flutter weinig built-in functionaliteit met als gevolg meer werk voor de programmeurs die meer code moesten schrijven ten opzichte van de andere meer ontwikkelde platformen. Fayzullaev omschrijft deze manier van coderen als ‘het opnieuw uitvinden van het wiel’. Ondanks het ontbreken van detail in dit onderzoek werd het wel gebruikt om kennis op te doen over de bouwstenen van Flutter.

Sebastian Faust schreef een Bachelorproef over het maken van grootschalige Flutter applicaties \autocite{Faust2020}. Het doel van dit onderzoek is andere applicatieontwikkelaars helpen wanneer zij voor dezelfde problemen komen te staan tijdens het schrijven van grote Flutter apps. Veel van de andere onderzoeken concludeerden dat Flutter beter is voor kleinschalige applicaties. Daarom was dit onderzoek interessant om op te nemen in deze literatuurstudie. Het vormde een benchmark onderzoek voor de schaalbaarheid van Flutter applicaties. Voor dit onderzoek schreef Faust een app met veel complexe features, tijdens dit proces documenteerde hij alle design keuzes en de problemen/obstakels die hieruit voortkwamen. Na het evalueren van elk obstakel kon hij altijd tot een optimale oplossing komen. Na de studie werden ondervindingen gebundeld en een set van richtlijnen opgesteld voor het ontwikkelen van grote Flutter applicaties. Richtlijnen en best practices waar developers zich best aan houden tijdens het ontwikkelingsproces. Deze werden gebundeld in een gids die later werd gepubliceerd en goed ontvangen werd door de Flutter community. Voor verdere ondersteuning van het onderzoek werd nog een interview afgenomen met een Flutter expert. De conclusie van de paper toonde de goede schaalbaarheid van Flutter apps aan maar dit onderzoek werd vooral gebruikt voor het oplossen van hinderblokken die Faust gedetailleerd onderzocht

Als laatste is het interessant om kort de nieuwe Flutter update aan te halen \autocite{Flutter2021}. De impact hiervan op dit onderzoek en de verschillen met vorige studies te duiden. Op woensdag 3 maart 2021 werd Flutter Engage georganiseerd, een live event gebracht door het Flutter team. Niet alleen het Flutter platform kreeg een upgrade maar ook de taal Dart werd verbeterd voor een vlottere en vertrouwelijke programmeer ervaring. De grootste Flutter upgrade was de overgang van een mobiel framework naar een portable framework. Oorspronkelijk werd Flutter alleen gecompileerd voor iOS en Android besturingssystemen maar door de upgrade zijn daar: Windows, macOS, Linux en Web bijgekomen. Echter valt dit buiten de scope van dit onderzoek. De grootste Dart upgrades worden hieronder opgelijst: 
\begin{itemize}
    \item Sound null safety is toegevoegd, zonder oude code te breken. Dart zal zeggen waar mogelijks gevaren zitten op null excepties.
    \item Smarter flow analysis
    \item Late variables
    \item Required named parameters
    \item DevTools upgrade
\end{itemize}

Ook werden in de keynote verschillende grote bedrijven zoals Ubuntu, Microsoft, Toyota… genoemd die in de toekomst nauw gaan samenwerken met Flutter. Deze samenwerkingen tonen de kracht, potentieel en groei van het platform. Ook werd in de Keynote het aantal open issues op Github aangehaald. Doordat het een opensource framework is kan iedereen die de broncode wil zien, deze gaan opzoeken op GitHub. Moest een fout stuk code, een simpeler te schrijven stuk code of zelfs als een goede bijdrage aan het platform gevonden worden kan hier een issue van gemaakt worden. Als Flutter deze verbetering heeft nagekeken en goedgekeurd wordt deze toegevoegd aan de broncode. Op deze manier sparen zichzelf werk uit en groeit het platform snel en volgens de wensen van de gebruikers. Het grootte aantal issues die momenteel open staan zien zij dan ook als een teken van groei. De vele gebruikers die Flutter willen beter maken is een teken dat er veel vertrouwen is in het platform. Door de updates en aanpassingen zullen veel API’s verouderd zijn. Flutter heeft dit echter goed geanticipeerd door het flutter fix commando toe te voegen. Deze zal de bestaande code doorlopen en tonen welke API’s verouderd zijn en in wat ze best vervangen worden.

\section{Waarom deze studie?}
\label{sec:waarom-deze-studie}
Cross-platform development heeft een snelgroeiende gebruikersbasis. De vele updates en verbeteringen aan de platformen maakt cross-platform development dan ook elke dag aantrekkelijker. Dit leidt tot meer onderzoek en studies binnen de sector. Tijdens de literatuurstudie bleek dat een groot aantal papers Flutter vergeleek met andere cross-platform ontwikkel tools. Dit soort papers moet de lezer helpen bij het maken van een keuze tussen deze tools. Door het verdelen van de aandacht over verschillende platformen, komt Flutter niet uitgebreid aanbod in dit soort papers. Voor 2020 waren de meeste onderzoeken van deze aard. Flutter leek toen nog niet stabiel en groot genoeg om als afzonderlijke tool onderzocht te worden.

Zoals uit voorgaande sectie blijkt, is Flutter meerdere malen onderzocht in 2020. In verschillende studies werd het tekort aan onderzoek voor 2020 aangehaald. Flutter is in het afgelopen jaar in een stroomversnelling geraakt op vlak van onderzoek. Dit is te danken aan de sterke stijging in gebruikersaantal. Het platform is nog steeds relatief jong maar de aantrekkingskracht ligt hem in de gebruikersbasis en de grote steun. Hoe meer gebruikers, hoe belangrijker het product in kwestie. Hoe belangrijker het product, hoe meer middelen hierin worden geïnvesteerd en dat lokt op zijn beurt weer meer gebruikers. Deze cirkel wordt alleen onderbroken als het platform in kwestie zijn aantrekkelijkheid verliest. Flutter lijkt hier geen last van te hebben en deze studie wil laten blijken waarom.

Dus kan geconcludeerd worden dat Flutter een interessant onderzoeksdomein is. Maar zoals bij elk onderzoek binnen de IT, zijn bevindingen al snel verouderd. De rappe evolutie binnen de sector doet elk product streven naar de beste versie van zichzelf. De competitie is hoog, dus moet performantie altijd voorop staan. Bij elke update of verbetering zijn voorgaande performantie studies verouderd en niet accuraat. Hierdoor zijn de studies, aangehaald in dit hoofdstuk, dan ook vaak zo recent mogelijk. Toch onderging Flutter net een grote release. Vele zaken werden verbeterd, onder andere de performantie van het platform. De voorgaande studies zijn weer verouderd en hier ligt een kans om opnieuw onderzoek te doen en beeld te schetsen van de huidige performantie van Flutter. Ook lijkt het interessant om een beeld te schetsen van de performantie van Flutter over tijd.

%%=============================================================================
%% Methodologie
%%=============================================================================

\chapter{\IfLanguageName{dutch}{Methodologie}{Methodology}}
\label{ch:methodologie}

%% TODO: Hoe ben je te werk gegaan? Verdeel je onderzoek in grote fasen, en
%% licht in elke fase toe welke stappen je gevolgd hebt. Verantwoord waarom je
%% op deze manier te werk gegaan bent. Je moet kunnen aantonen dat je de best
%% mogelijke manier toegepast hebt om een antwoord te vinden op de
%% onderzoeksvraag.

Om een antwoord te kunnen bieden op zowel de primaire, als de deelonderzoeksvragen, werd een onderzoek verricht. Dit onderzoek bestaat uit twee grote delen. Het eerste deel is de literatuurstudie die in vorig hoofdstuk beschreven werd. Het tweede deel is een experiment waaruit ondervindingen werden geanalyseerd en samengevat in volgende hoofdstukken. Om inzicht te krijgen in de opzet van het experiment wordt hieronder de opbouw hiervan uitgelegd. Ook wordt elk onderzoekscriterium toegelicht. 

\section{Opzet Experiment}
\label{sec:opzet-experiment}
%TODO maak duidelijk dat lijnen code, complexiteit... allemaal onder een hoofdstuk vallen
Zoals reeds vermeld, valt dit experiment onder het luik vergelijkende studie. Het gaat twee ontwikkeltechnieken met elkaar vergelijken. Om deze technieken op een correcte manier te vergelijken werden verschillende applicaties ontwikkeld.

\textbf{Applicaties}\\
Voor het experiment werden vier applicaties ontwikkeld. Twee applicaties met de Flutter stack en twee met de Android stack. Het eerste paar applicaties dat vergeleken werd waren de Hello world applicaties (zie appendix). Deze werden gebruikt om twee van de drie performantie criteria te onderzoeken. Voor het criterium CPU gebruik en de andere criteria werd een tweede paar applicaties ontwikkeld, verder onderzoeksapplicatie. Het ontwikkelproces van dit tweede paar applicaties werd aanschouwt als onderzoek.

\newpage
\textbf{Omgeving}\\
De applicaties werden ontwikkeld in de Android Studio Integrated Development Enviroment (IDE). Android Studio biedt de snelste tools voor het bouwen van apps op elk type Android-apparaat. Daarnaast werd tevens voor Android Studio gekozen gezien de uitgebreide mogelijkheden, het beperkt houden van ontwikkeltools alsook het gegeven dat deze applicatie gratis te gebruiken is. Een andere IDE die voor Android ontwikkeling gebruikt kan worden is IntelliJ IDEA. Voor Flutter ontwikkeling wordt de Visual Studio Code IDE ook veel gebruikt. Een vergelijking tussen de drie IDE’s valt echter buiten de scope van dit onderzoek. 

\textbf{Talen}\\
De Android applicatie werd geschreven in de Kotlin taal. Kotlin is een gratis, open source programmeertaal ontworpen voor Java Virtual Machine en Android. De Flutter applicatie werd ontwikkeld worden in de Dart taal. Dit is een onafhankelijke taal maar ze wordt vooral gebruikt voor de ontwikkeling van Flutter applicaties.

\textbf{Lijnen code}\\
Tijdens de uitwerking van elk onderzoekscriteria werd rekening gehouden met een aantal sub-criteria. Op basis van deze sub-criteria werd een conclusie gevormd over de toegankelijkheid van beide frameworks. Eerst werd gekeken naar het aantal geschreven lijnen code. Het aantal lijnen code staat niet garant voor een betere sensatie van de app. Doch is het interessant om te kijken welke zaken bij het ene platform al dan niet uitgebreider dienen geïmplementeerd te worden om hetzelfde resultaat te bekomen. 

\textbf{Code complexiteit}\\
Vervolgens werd gekeken naar de complexiteit van de geschreven code. Dit was interessant om een beeld te krijgen van de toegankelijkheid van de taal. Hier wordt een onderscheid gemaakt tussen de syntax van de taal, de gebruikte built-in functionaliteit en de documentatie hiervan. Code complexiteit gaat in vele gevallen gepaard met het aantal lijnen code. Eerst en vooral is het niet altijd beter om alles in één beknopte lijn code te schrijven. Het is zo dat bij het schrijven van applicaties, vaak samengewerkt wordt met anderen aan dezelfde code. Complexe code is moeilijker om lezen en kan soms verkeerd geïnterpreteerd worden. Het is dus niet verkeerd om soms een extra lijn code te schrijven voor de complexiteit te verminderen. Echter moet hier een gezonde middenweg in gezocht worden aangezien veel lijnen code ook voor een complexe codebase kan zorgen.

\textbf{Libraries of packages}\\
Libraries in Android en packages in Flutter zijn een belangrijk en extreem krachtig aspect van ontwikkeling. Een library of package (verder library), is een verzameling van code die bepaalde functionaliteit reeds uitgewerkt heeft en dus zorgt voor snelle herbruikbare code. Omdat dit de snelheid van ontwikkeling kan beïnvloeden en tevens de robuustheid van de applicatie kan verbeteren, is het belangrijk om hier ook aandacht aan te besteden. Libraries kunnen de robuustheid van een applicatie verbeteren gezien de meeste libraries vaak open source zijn. Op deze manier worden nieuwe functionaliteiten toegevoegd die nodig blijken in verschillende use cases.
\newpage
Het gebruik van libraries kan drastisch helpen met de code complexiteit en het aantal geschreven lijnen code. Libraries nemen vaak een deel van de complexe code op zich. Dit leidt tot minder en simpelere code. Echter is het niet altijd gemakkelijk om robuuste en goed ondersteunde libraries te vinden. Bij Android gaat dit al vlotter aangezien tijd heeft doen blijken welke libraries goed ondersteund werden en dus ook veel gekozen werden.


\section{Onderzoekscriteria}
\label{sec:onderzoekscriteria}
%TODO uitleg over werking van hoofdstukken
In volgende hoofdstukken worden de onderzoekscriteria uitgewerkt. Hieronder wordt uitgelegd hoe elk van deze criteria onderzocht werd.

Onder het luik performantie vallen drie onderzoekscriteria. Deze worden ook wel de performantie criteria genoemd. Eerst hebben we de grootte van de uitvoeringsbestanden. Dit werd onderzocht aan de hand van de Hello World applicaties. Hierbij werden de groottes van de apk’s van beide apps vergeleken. Vervolgens werd de opstartsnelheid van de twee Hello world apps onderzocht. Hierbij werden beide applicaties x aantal keer geopend en werd telkens de duur van de opstartprocedure genoteerd. Hiervan werd uiteindelijk het gemiddelde genomen en deze bevindingen voor zowel Android als Flutter werden dan met elkaar vergeleken. Daarna werd het CPU gebruik van beide applicaties onderzocht. Hierbij werd gebruik gemaakt van de onderzoeksapplicatie.

Vervolgens werd de creatie van views onderzocht.

Het volgende onderzochte criterium was het uit voeren van asynchroon taken binnen elk framework. Hierbij werden de verschillende manieren van asynchroon werken aangehaald. De voor- en nadelen van deze asynchrone ontwikkelmanieren werden opgesomd en uiteindelijk werden met elkaar vergeleken. 

Security

Daarna werd gekeken naar de beschikbare libraries en code complexiteit

Beschikbare tools

\section{Gebruikte hardware}
\label{sec:hardware}
%TODO hoofdstuk schrijven
%%=============================================================================
%% Performantie
%%=============================================================================


\chapter{\IfLanguageName{dutch}{Performantie}{Performance}}
\label{ch:performantie}
Dit hoofdstuk zal het performantie aspect van de applicaties behandelen. Ondanks het feit dat nieuwe smartphones alsmaar performanter worden, moet nog steeds belang gehecht worden aan de performantie van een applicatie tijdens de ontwikkeling. Hetgeen evenredig stijgt met de performantie van nieuwe smartphones, zijn de gebruikersverwachtingen van de apps. Het is belangrijk dat app ontwikkelaars een applicatie zo schrijven dat deze de beste performantie biedt aan de gebruiker. Elke app ontwikkelaar heeft namelijk als hoofddoel het ontwikkelen van een app die veel gebruikt wordt en een goede ervaring bezorgd aan de gebruiker. Wanneer een app niet voldoet aan de gebruikerseisen zal deze stoppen met de app te gebruiken, deze verwijderen en misschien zelf een negatieve review achterlaten. Wat op zijn beurt leidt tot minder gebruikers en minder downloads. 

Er zijn verschillende aspecten die de performantie van een app beïnvloeden. Deze kunnen onderverdeeld worden in verschillende categorieën. In dit onderzoek werd alleen gefocust op de app performantie. Hieronder vallen laadsnelheid, batterij verbruik, geheugen gebruik, grootte van het uitvoeringsbestand, geheugen gebruik van de app in achtergrond… Dit onderzoek spitst zich toe op opstartsnelheid, grootte van de uitvoeringsbestanden en CPU gebruik. Ook moet rekening gehouden worden met de samenhang tussen performantie en gebruikte hardware. De gebruikte hardware voor dit onderzoek werd vermeld in vorige hoofdstuk. Het is dus belangrijk om een performante app te ontwikkelen maar het gebruikte framework zit hier ook voor een deel tussen. Het doel van dit hoofdstuk is het vergelijken van deze onderliggende verschillen. Deze worden hieronder kort toegelicht.

\section{De grootte van de uitvoeringsbestanden}
\label{sec:uitvoeringsbestanden}
Het eerste performantie aspect dat onderzocht werd was de grootte van het uitvoeringsbestand. Het doel bij applicatieontwikkeling is om een uitvoeringsbestand zo klein mogelijk te houden, zodat de gebruiker geheugen kan besparen op zijn toestel. Het is immers zo dat de prijs van high-end smartphones vandaag de dag vrij steil is. De consument kan bij het aanschaffen van een nieuw toestel vaak kiezen voor een bepaalde hoeveelheid opslag. Wanneer voor een toestel gekozen wordt met een vrij beperkte opslagcapaciteit en de gebruiker een aantal grote apps installeert, zal de opslagcapaciteit van het toestel snel volledig ingenomen zijn. Dit kan tot gevolg hebben dat de gebruiker bepaalde grotere applicaties zal gaan verwijderen om ruimte te creëren. Dit is één van de redenen waarom om een app zo klein mogelijk te houden en ondertussen de hoogste performantie te garanderen. 

Een andere reden is de maximale toegestane grootte van de uitvoeringsbestanden opgelegd door de verschillende app stores. Om een applicatie op de Google Play Store te kunnen beschikbaar stellen, moet deze voldoen aan meerdere eisen. Een van deze eisen is dat het uitvoeringsbestand van de app maximaal 100MB bedraagt. Het doel van een applicatie beschikbaar te stellen op de Play Store is immers om een zo groot mogelijk doelpubliek aan te spreken. Wanneer echter niet aan de eisen van de Play Store voldaan wordt, zal dit doelpubliek niet bereikt worden. In dit hoofdstuk zal gekeken worden naar de grootte van de uitvoeringsbestanden van respectievelijk de Flutter applicatie alsook de Android applicatie.

\subsection{Opzet}
Een uitvoeringsbestand, ook wel gekend als een executable, van een Android app kan verschillende bestandsformaten zijn. De twee meest gebruikte formaten zijn Android Package (APK) en App Bundle. Deze bestanden worden gebruikt om een Android applicatie uit te voeren. Voor dit onderzoek werd gefocust op het APK bestandsformaat. Een APK is een map of verzameling van bestanden. Al deze bestanden samen zorgen ervoor dat een applicatie kan worden uitgevoerd. Zo werden voor dit onderzoek twee APK's gegenereerd van de Hello world applicaties. De eerste vanuit de native Android codebase en de tweede van de Flutter codebase. Voor de meest bruikbare resultaten werd gekozen om voor de applicaties een release build te maken. Elke build is een verzameling van een aantal regels die worden toegepast op de code wanneer deze compileert. De release build is een verzameling van regels die de app zo compact mogelijk maakt. Dit gebeurt door het opruimen en optimaliseren van code. De resultaten van het onderzoek werden bekomen gebruik makend van de apkanalyzer, een ingebouwde Android Studio tool die de APK ontleed en de grootte van elk van deze delen weergeeft. 

Tabel \ref{table:maatstafUitvoeringsbestand} biedt inzicht in de gebruikte maatstaf voor dit deel van het onderzoek.
\begin{table}
    \begin{center}
        \caption{Gebruikte maatstaf voor de grootte van de uitvoeringsbestanden}
        \label{table:maatstafUitvoeringsbestand}
        \begin{tabular}{ |l|c|c| }
            \hline
            Bit & / & Binair getal, 1 of 0\\
            \hline
            Byte & B & 8 Bit \\ 
            \hline
            Kilobyte & KB &  \[10^{3}\]B\\ 
            \hline
            Megabyte & MB & \[10^{6}\]B \\ 
            \hline
        \end{tabular}
    \end{center}
\end{table}

\newpage

\textbf{Android}\\
In Android wordt een release build gemaakt aan de hand van volgende lijnen code in het app build.gradle bestand \ref{lst:label}.
\begin{lstlisting} [caption={Android build.gradle (app)},  label={lst:label}]
release {
    minifyEnabled true
    shrinkResources true
    proguardFiles getDefaultProguardFile
    ('proguard-android-optimize.txt'), 'proguard-rules.pro'
}
\end{lstlisting} 

\textbf{Flutter}\\
Voor een ideale Flutter release build wordt best gebruik gemaakt van onderstaande terminal commando’s.\\
\textit{flutter clean}\\
\textit{flutter build apk --split-per-abi}\\
Het flutter clean commando zal het project kleiner maken door het verwijderen van de build en .dart-tool mappen. Het commando daarna maakt een release APK per ABI. De APK’s worden gesplitst per ABI omdat dit het uitvoeringsbestand aanzienlijk verkleint. Verdere uitleg over ABI kan terug gevonden worden in hoofdstuk \ref{ch:appendix} Appendix.

Als laatste werd een opsomming gemaakt van de grootte van Flutter uitvoeringsbestanden over tijd. Dit moet een inzicht bieden in de evolutie van Flutter.

\subsection{Resultaten}
\textbf{Android}\\
Een Android app heeft geen minimum grootte voor het uitvoeringsbestand. Echter als de app op de Play Store dient gezet te worden moet deze minimum 7KB bedragen. Wanneer een nieuw Native Kotlin Android project wordt opgezet met één activity, zal dit automatisch een Hello World applicatie zijn. Als eerste stap in het onderzoek werd gekeken naar de omvang van deze APK. Volgens apkanalyzer was deze APK 3.2 MB in omvang en 2.6 MB voor de download. Vervolgens werd gekeken naar een verkleinde versie van de codebase waaruit de testing directories verwijderd werden samen met de bijhorende dependencies. Vervolgens werd minifyEnabled als ook shrinkResources op true gezet. Dit leidt tot een APK van 1.6 MB met een downloadgrootte van 1010.1KB.
Voor een derde en finale versie van de APK werd de codebase nog kleiner gemaakt, met als gevolg dat niet meer werd gewerkt volgens de Android best practices. Tabel \ref{table:androidUitvoerinsbestanden} bevat de resultaten van dit onderzoek. De code voor deze Hello world applicatie kan terug gevonden worden op GitHub repository \autocite{DePauw2021}.

\begin{table}
    \begin{center}
        \caption{Grootte van de uitvoeringsbestanden van de Hello world app in Android \autocite{DePauw2021}}
        \label{table:androidUitvoerinsbestanden}
    \begin{tabular}{ | l | m{3cm} | m{3cm} | }
        \hline
        & \textbf{APK size} & \textbf{Download Size}\\
        \hline
        META-INF & 3 KB & 3 KB\\ 
        \hline
        res & 123,7 Kb & 117 KB\\ 
        \hline
        AndroidManifest,xml & 721 B & 721 B\\ 
        \hline
        classes.dex & 293,9 KB & 293,1 KB\\ 
        \hline
        Resources.arsc & 222,6 KB & 51,7 KB\\ 
        \hline
        Kotlin & 9,1 KB & 9 KB\\ 
        \hline
        \hline
        \textbf{Total} & \textbf{693,2 KB} & \textbf{476,1 KB} \\ 
        \hline
    \end{tabular}
\end{center}
\end{table}

\textbf{Flutter}\\
Flutter maakt gebruik van de Flutter Engine (zie hoofdstuk \ref{ch:appendix} Appendix). Deze engine maakt deel uit van de APK en is nodig voor het gebruiken van een Flutter app. Deze engine bevat het gehele framework en is daarom een aantal megabyte groot. Dit maakt de Flutter APK al direct meerdere megabytes in omvang. Voor het eerste deel van het onderzoek werd gekeken naar een standaard versie van de Hello world template. Hierbij bevatte de main.dart file 26 lijnen code. De niet gesplitste ABI versie was 15.5 MB groot terwijl de gesplitste APK 5.1 MB bedroeg voor de gewone ARM en 5.5 MB voor de ARM64. Vervolgens werd de applicatie herschreven met als doel een zo klein mogelijke APK te genereren. Hierbij was het main.dart bestand 12 lijnen code na het uitvoeren van een code format. De APK versie waarbij niet gesplitst werd per ABI was 14 MB groot. De resultaten van de APK gesplitst op ABI staan vermeld in tabel \ref{table:flutterUitvoeringsbestanden}. De code voor dit onderzoek kan terug gevonden worden op de GitHub repository \autocite{DePauw2021}.

\begin{table}
    \begin{center}
        \caption{Grootte van de uitvoeringsbestanden van de Hello world app in Flutter \autocite{DePauw2021}}
        \label{table:flutterUitvoeringsbestanden}
        \begin{tabular}{ | l | m{2.5cm} | m{2.5cm} | m{2.5cm} | m{2.5cm} | }
            \hline
            & \multicolumn{2}{|c|}{\textbf{ARM}} & \multicolumn{2}{|c|}{\textbf{ARM-64}}\\
            \hline
            & \textbf{APK Size} & \textbf{Download size} & \textbf{Apk Size} & \textbf{Download Size}\\ 
            \hline
            lib & 4,2 MB & 4,2 MB & 4,7 MB & 4,6 MB\\ 
            \hline
            assets & 183,4 KB & 183 KB & 183,4 KB & 183 KB\\ 
            \hline
            META-INF & 7,9 KB & 7,6 KB & 7,9 KB & 7,6 KB\\ 
            \hline
            res & 6,2 KB & 5,9 KB & 6,2 KB & 5,9 KB\\ 
            \hline
            AndroidManifest.xml & 1021 B & 1021 B & 1022 B & 1022 B\\ 
            \hline
            classes.dex & 121,7 KB & 121,3 KB & 121,7 KB & 121,3 KB\\ 
            \hline
            Resources.arsc & 22,6 KB & 3,8 KB & 22,6 KB & 3,8 KB\\ 
            \hline
            Kotlin & 9,7 KB & 9,7 KB & 9,7 KB & 9,7 KB\\ 
            \hline
            \hline
            \textbf{Total} &  \textbf{4,6 MB} &  \textbf{4,5 MB} &  \textbf{5 MB} &  \textbf{5 MB}\\ 
            \hline
        \end{tabular}
    \end{center}
\end{table}

Op de Flutter site \autocite{Flutter} staat een korte omschrijving van de minimale downloadgrootte gemeten van een Flutter app (geen materiële componenten, slechts een enkele Center-widget, gebouwd met flutter build apk --split-per-abi), gebundeld en gecomprimeerd als een release APK. Deze minimale downloadgrootte werd in 2018 verschillende malen verkleint en opnieuw gemeten. Tabel \ref{table:flutterUitvoeringsbestandenOverTijd} bundelt de resultaten van deze metingen.

\begin{table}
    \begin{center}
        \caption{Grootte van de uitvoeringsbestanden in Flutter over tijd}
        \label{table:flutterUitvoeringsbestandenOverTijd}
        \begin{tabular}{ |l|c| }
            \hline
            \textbf{Tijd} & \textbf{Download Size}\\
            \hline
            Maart 2018 & 4,06 MB\\ 
            \hline
            Augustus 2018 & 4,20 MB\\ 
            \hline
            Begin Oktober 2018 & 4,28 MB\\ 
            \hline
            Eind Oktober 2018 & 4,48 MB\\ 
            \hline
            November 2018 & 4,70 MB\\ 
            \hline
            December 2018 & 6,70 MB\\ 
            \hline
        \end{tabular}
    \end{center}
\end{table}
\section{De opstartsnelheid van de applicatie}
\label{sec:opstartsnelheid}
Een andere veel voorkomende reden waarom gebruikers applicaties slecht beoordelen of verwijderen is een trage applicatie. Onderzoek toont aan dat gebruikers verwachten dat een applicatie binnen maximaal drie seconden opstart. Met opstartsnelheid wordt de tijd bedoelt tussen het drukken op het app icoon en het te zien krijgen van een view. Hierbij moet rekening gehouden worden met de in te laden data. Het inladen van data heeft in veel gevallen weinig te maken met app performantie. In veel gevallen zal dit te maken hebben met de aangesproken API. Een onderzoek naar de respons tijd van een API valt echter buiten het bestek van dit onderzoek. Het verschil tussen het inladen van data en het opstarten van de app kan (meestal) gezien worden aan de hand van een laad icoon. Als de app een API aanspreekt voor data zal (in de meeste gevallen) een laad icoon getoond worden terwijl tijdens de opstart procedure de app in geheugen wordt geladen en dus nog niets getoond kan worden.

Een performante app betekent in vele gevallen een snelle app, dus kan een trage opstartsnelheid leiden tot frustratie. Aangezien elke ontwikkelaar mikt op een zo goed mogelijke user experience (UX), is het interessant om ook de opstartsnelheid van een applicatie te onderzoeken. De vraag die hier beantwoord werd was volgende 'Wat is de impact van beide ontwikkelingstechnieken op de opstartsnelheid?'.

\subsection{Opzet}
Om het verschil in opstartsnelheid tussen beide platformen te testen, werd gebruik gemaakt van de Hello world applicaties. Door de kleine omvang van de applicaties, waren de verschillen in tijd beperkt. Ook werd rekening gehouden met het feit dat deze opstart tijden variabel zijn, daarom werd het onderzoek gevoerd aan de hand van een groot aantal iteraties. Hieruit konden minimum, maximum, mediaan, gemiddelde en standaard afwijking van de opstarttijden berekend worden per applicatie, die het mogelijk maakten om beide platformen met elkaar te vergelijken. 

Hierbij dient wel vermeld te worden dat beide applicaties vanaf nul gestart werden. Het is namelijk zo dat applicaties uit drie verschillende toestanden gestart kunnen worden. De Engelse termen voor deze drie opstartprocedures zijn volgende: cold start, warm start en hot start (zie hoofdstuk \ref{ch:appendix} Appendix). In dit onderzoek werd gebruik gemaakt van de cold start procedure. Deze procedure is degene die de grootste uitdaging vormt in het kader van het minimaliseren van de opstarttijd.

In Android 4.4 (API-niveau 19) en hoger bevat logcat een regel uitvoer met de waarde 'Displayed'. Deze waarde vertegenwoordigt de hoeveelheid tijd die is verstreken tussen het starten van het proces en het voltooien van het tekenen van de bijbehorende activiteit op het scherm. Dit werd gebruikt voor het bepalen van de tijd in Android.

In Flutter werd het commando flutter run --trace-startup --profile gebruikt. De trace-uitvoer wordt opgeslagen als een JSON-bestand met de naam start\_up\_info.json onder de build directory van het Flutter-project. De uitvoer geeft de verstreken tijd weer vanaf het opstarten van de app tot deze trace events. Echter toont de uitvoer van dit commando ook de tijd nodig voor Flutter om de app vanaf nul op te starten tot de eerste pixel op het scherm wordt getekend.

\subsection{Resultaten}
Zie tabel \ref{table:opstartsnelheid} voor de resultaten van het onderzoek.
\begin{table}
    \begin{center}
        \caption{Opstartsnelheden van de Android en Flutter Hello world apps. \autocite{DePauw2021}}
        \label{table:opstartsnelheid}
        \begin{tabular}{ | l | l | l |}
            \hline
            & \textbf{Android} & \textbf{Flutter}\\
            \hline
            \textbf{Hoogst} & 985ms & 848ms\\
            \hline
            \textbf{Laagst} & 582ms & 293ms\\
            \hline
            \textbf{Mediaan} & 649ms & 416,5ms\\
            \hline
            \textbf{Gemiddeld} & 681,31ms & 454,38ms\\
            \hline
            \textbf{Standaard afwijking} & 89,6450571ms & 126,5891134ms\\
            \hline
        \end{tabular}
    \end{center}
\end{table}

\section{Discussie}
Uit onderzoek van Bracke \autocite{Bracke2020} bleek dat in 2020 de opstartsnelheid van native Android apps gemiddeld 146,7ms lager was dan Flutter apps. Maar aan de andere kant dat de Flutter apps een stabielere opstarttijd ondervonden. Dit is volledig het tegenovergestelde met wat dit onderzoek aantoont. De resultaten zijn volledig anders wat doet vermoeden dat de omgevingsvariabelen niet hetzelfde zullen geweest zijn. Een andere mogelijkheid is de onderliggende hardware die hier een te grote rol speelde.
\section{Het CPU gebruik van de applicatie}
\label{sec:cpu}
Een Central Processing Unit of kortweg CPU is het als het ware het brein van een computer. Het is een stuk hardware dat aan de hand van een aantal basisoperaties, programma’s uitvoert. Om deze operaties performant uit te voeren heeft de CPU echter nood aan een vaste, degelijke bron van energie. In het geval van mobiele apparaten, zoals een laptop of een smartphone, is deze energiebron de batterij van het toestel. Vandaag de dag moeten, en zijn, smartphones vrij tot zeer compact. Dit heeft echter ook een nadeel. Door de compactheid van de toestellen, is de plek voor de batterij ook eerder beperkt. Om de capaciteit van de batterij zo goed mogelijk te benutten, moeten de processen die op de toestellen draaien dus eerder performant zijn. Hiermee wordt bedoeld dat ze energiezuinig moeten zijn zonder snelheid in te leveren. 

Een app die veel CPU werk vereist zal dus meer energie verbruiken, de batterij sneller doen leeglopen en kan zorgen voor oververhitting van de batterij. Een probleem dat voorkomt wanneer de app meer resources gebruikt als nodig.

Aangezien het beheer van de batterijduur een dagelijkse ervaring is voor velen, wil men zo weinig mogelijk batterij verspelen aan korte app interacties. Gebruikers die merken dat een app te veel vraagt van de CPU zullen deze app dan ook sneller verwijderen. Daarom was het interessant om te onderzoeken of de onderliggende frameworks hier een aandeel in hebben. 

\subsection{Opzet}
Voor het onderzoek naar CPU gebruik werden de onderzoeksapplicaties vergeleken. Deze applicaties zijn omvangrijk en bevatten meerdere features die CPU intensieve taken uitvoeren. 

Het CPU gebruik wordt uitgedrukt in procent. Hoe hoger dit getal hoe meer CPU de app in gebruik nam. De gehele app werd doorlopen en het gemiddelde CPU gebruik werd vergeleken. Uitschieters werden ook opgenomen in de resultaten voor het maken van vergelijkingen omtrent CPU intensieve components.

Voor het onderzoek naar CPU gebruik werd voor Android gebruik gemaakt van de Android Profiler. Een krachtige monitoring tool die het mogelijk maakt om live CPU, geheugen, netwerk en batterij resources op te volgen. Meer specifiek werd gekeken naar de CPU Profiler. Voor Flutter werd gebruik gemaakt van de Flutter DevTools. Een uitgebreide tool die in Flutter 2.0 een grote update kende. Hiermee kunnen live zaken zoals logging, geheugen gebruik, CPU gebruik, netwerk requests… worden opgevolgd.

\subsection{Resultaten}
Zie tabel \ref{table:cpu}
\begin{table}
    \begin{center}
        \caption{Opstartsnelheid}
        \label{table:cpu}
        \begin{tabular}{ | l | l | l |}
            \hline
            & \textbf{Android} & \textbf{Flutter}\\
            \hline
            \textbf{Hoogst} & \% & \%\\
            \hline
            \textbf{Laagst} & \% & \%\\
            \hline
            \textbf{Mediaan} & \% & \%\\
            \hline
            \textbf{Gemiddeld} & \% & \%\\
            \hline
            \textbf{Standaard afwijking} & \% & \%\\
            \hline
        \end{tabular}
    \end{center}
\end{table}


Tabel \ref{table:cpu} laat zien dat Flutter en Android in dit criterium niet veel van elkaar verschillen. De Flutter applicatie had een lager maximaal CPU-gebruik, maar de native Android had een lagere gemiddelde waarde en beide hadden dezelfde laagste waarde. De Android-applicaties hadden over het algemeen in het begin hoge CPU-prestaties.
\newpage

\section{Conclusie}
\label{sec:perfConclusie}

Als eerste werden de uitvoeringsbestanden onderzocht. Wat uit deze resultaten kan geconcludeerd worden is dat een duidelijk verschil bestaat tussen enerzijds de native APK’s en anderzijds de cross-platform APK’s. Het is namelijk zo dat native APK’s aanzienlijk kleiner zijn in vergelijking met hun cross-platform tegenhangers. Dit komt door de Flutter Engine die mee moet verpakt worden in de APK's. Een minimale Flutter APK is 4,6 MB groot wat 3,9 MB groter is als een minimale Android APK. Echter bij grotere APK’s wordt het verschil tussen Android en Flutter eerder klein en insignificant. Vervolgens werden de opstartsnelheden vergeleken. Hier bleek Flutter een voordeel te hebben over Android, echter was dit verschil klein en amper waarneembaar voor een gebruiker. De Flutter app was gemiddeld 227ms sneller dan de Android app. Het testen op CPU-verschillen in de applicaties toonde aan dat er niet veel verschil was in CPU-gebruik tussen native Android builds en Flutter builds. Dit zou het gevolg kunnen zijn van het gebruik van een kleine applicatie. Het experiment zou andere resultaten kunnen opleveren als dit deel van het onderzoek op een grotere applicatie werd uitgevoerd. Dus uitgesloten de APK omvang zijn deze performantie criteria van de twee platformen gelijkaardig te noemen.
%%=============================================================================
%% Asynchroon
%%=============================================================================


\chapter{\IfLanguageName{dutch}{Asynchroon Werken}{Asynchronous Tasks}}
\label{ch:asynchroon}
Bij elk stuk code maakt de ontwikkelaar een keuze over het achterliggende uitvoeringsproces, moet het stuk code parallel of sequentieel verlopen met de andere stukken code. Deze keuze is vaak onbewust maar wel belangrijk voor de UX van een applicatie. 

Wanneer gekozen wordt om synchroon of sequentieel te werken, wordt gewacht op de uitkomst van een proces vooraleer een nieuw proces kan gestart worden. Hierbij wordt elk proces afgewerkt en enkel wanneer een bepaald proces afgewerkt is, kan een ander proces gestart worden. 

Wanneer gekozen wordt om asynchroon of parallel te werken, wacht men niet op de uitkomst van het andere proces maar start het nieuwe proces ongeacht. Hierbij zijn dus meerdere processen op hetzelfde moment actief. Dit heeft als gevolg dat meerdere verschillende problemen aangepakt kunnen worden. Hier worden verschillende processen afgehandeld op verschillende threads. Een thread kan beschouwd worden als een rijstrook op de autosnelweg, er zijn verschillende threads waarop verschillende processen tegelijkertijd kunnen uitgevoerd worden, net zoals op de verschillende rijstroken meerdere auto’s tegelijkertijd kunnen rijden. 

Voor bepaalde taken zoals het ophalen en persisteren van data wordt vaak gekozen voor asynchrone taken. Een scherm moet elke 16 milliseconden worden geüpdatet om de user een vlotte ervaring te geven. Dit kan alleen gebeuren als de UI thread niet geblokkeerd wordt door zware en langdurige taken. In Android is de main thread gelijk aan de UI thread.

Op zowel Android als Flutter zijn verschillende manieren aanwezig om taken asynchroon uit te voeren. In dit hoofdstuk worden de meest gebruikte manieren aangehaald.
 


\section{Opzet}
\subsection{Android}
\underline{Coroutines}\\
Een coroutine is iets dat in Android kan gebruikt worden om het schrijven van asynchrone code te vereenvoudigen. Op Android helpen coroutines om langdurige taken af te handelen die anders de main thread zouden blokkeren. Coroutines zorgen er in dit geval voor dat de app niet blokkeert of niet meer reageert. Een use case waarvoor coroutines vaak gebruikt worden is bijvoorbeeld wanneer er een API call uitgevoerd wordt. Op deze manier kan deze call in de achtergrond afgehandelt worden zonder dat de rest van de applicatie er op dient te wachten alvorens andere taken te kunnen vervullen. 

\underline{Java Thread}\\
De Java Virtual Machine (JVM) maakt het mogelijk dat verschillende threads tegelijkertijd lopen. Elk van deze threads heeft een prioriteit en op basis daarvan wordt een thread al dan niet uitgevoerd voor een andere thread. Wanneer een JVM opstart, is er gebruikelijk één thread aanwezig die de main method aanroept. De JVM blijft threads uitvoeren tot bijvoorbeeld de exit methode van de Runtime class aangeroepen wordt en de security manager groen licht gegeven heeft om de exit methode uit te voeren. 

\underline{AsyncTasks}\\
Deprecated 

\underline{Android services}\\
Android service is een component dat gebruikt wordt om processen uit te voeren in de achtergrond zoals muziek afspelen, netwerk transacties behandelen,.. Het proces blijft oneindig in de achtergrond lopen zelfs als de applicatie afgesloten wordt.
De keuze tussen een thread of een service is afhankelijk van de use case van de applicatie. Als er processen moeten gebeuren buiten de main thread, maar enkel wanneer de applicatie actief is en de gebruiker er dus met interageert, is het aangeraden om een nieuwe thread te maken en dus geen service te gebruiken. Zo kan het bijvoorbeeld nodig zijn dat er muziek afgespeeld moet worden terwijl de applicatie actief is, dit is op te lossen aan de hand van een thread waardoor het een slecht idee zou zijn om hiervoor een service te gebruiken. 

\underline{Callbacks}\\


\underline{EventBus}\\
EventBus is een open source library voor Android en Java die gebruik maakt van een publisher/subscriber patroon. Hierbij is het idee dat een publisher een event (of proces) uitstuur naar een Event Bus via een post() call. Deze Event Bus zal vervolgens dit event naar de subscribers ervan uitsturen waardoor zij kunnen reageren van zodra er iets veranderd is gebruikmakend van de onEvent() call. De voordelen van EventBus zijn bijvorrbeeld de eenvoudige communicatie tussen componenten, de goede performantie bij gebruik van activities, fragments en background threads en het feit dat het reeds bewezen is in de praktijk. 

\underline{Reactive programming (RxJava)}\\


\underline{Chanel}\\


\subsection{Flutter}

\section{Resultaten}
%%=============================================================================
%% API's
%%=============================================================================


\chapter{\IfLanguageName{dutch}{Netwerk calls}{Network calls}}
\label{ch:api}
Wanneer ontwikkelaars de term API horen, weten zij direct waarover het gaat. Een Application Programming Interface, of dus kortweg een API, is software die het toelaat om verschillende applicaties of applicatielagen met elkaar te laten communiceren. Een van de primaire use cases waarvoor een API gebruikt wordt in de mobiele applicatie wereld, is die om de zogenaamde API calls uit te voeren. Het doel van deze API calls zijn om vanuit de mobiele applicatie, de API aan te spreken om data op te halen die vervolgens getoond kan worden. Een API zorgt er in dit geval voor dat een mobiele applicatie in connectie staat met de cloud. Naast het ophalen van data kan er vaak ook nieuwe data naar de API gestuurd worden, of kan bestaande data zelfs aangepast worden.

\section{Opzet}
Voor dit en de volgende experimenten, werd een nieuwe applicatie uitgewerkt voor beide platformen. Deze applicatie bevat voorts alle nodige functionaliteit waaraan dient voldaan te worden om de experimenten tot een goed eind te brengen. Voor dit specifieke experiment, namelijk degene waarvoor gebruik gemaakt wordt van een online API, werd ook functionaliteit uitgewerkt. Er werd gekozen om een online API te gebruiken die duidelijk en goed ondersteund was. Eén van de bekendste en gebruiksvriendelijke API’s is de PokéApi. Deze API, die informatie omtrent allerlei Pokémon teruggeeft, is duidelijk ondersteunt en gedocumenteerd waardoor hij perfect is voor dit experiment. De documentatie omtrent de PokéApi is te lezen op een overzichtelijke site die gebruik maakt van een RESTful interface. 

Om dit experiment uit te voeren, werd gebruik gemaakt van zogenaamde data klassen. Deze klassen worden gebruikt om de verschillende objecten op te bouwen die terugkomen van de API. Aan de hand van bijvoorbeeld een framework, kan dan gebruik gemaakt worden van deze objecten om de data uit te lezen en te gebruiken in de applicatie zelf. Enkele van de libraries die hiervoor in Android veel, en makkelijk te gebruiken zijn, zijn Retrofit, Moshi, Chuck, Glide en Volley. Voor Flutter wordt vooral gebruik gemaakt van de HTTP Package library.	

\section{Resultaten}
\section{Conclusie}
%%=============================================================================
%% Security
%%=============================================================================


\chapter{\IfLanguageName{dutch}{Security}{Security}}
\label{ch:security}

\lipsum[1-2]
%%=============================================================================
%% Complexiteit
%%=============================================================================


\chapter{\IfLanguageName{dutch}{Code Complexiteit}{Code Complexity}}
\label{ch:complexiteit}

\lipsum[1-2]

%%=============================================================================
%% Tooling
%%=============================================================================

\chapter{\IfLanguageName{dutch}{Tooling}{Tooling}}
\label{ch:tooling}
Een besturingssysteem is een interface tussen de gebruiker en de hardwarecomponenten. Het voert verschillende taken uit. Een besturingssysteem biedt de gebruiker een Graphical User Interface (GUI) of Command Line Interface (CLI) aan om taken uit te voeren. Sommige besturingssystemen bieden alleen een CLI of GUI aan, terwijl andere zowel GUI als CLI aanbieden. Een GUI bestaat uit bedieningselementen of widgets om met de computer te communiceren. Aan de andere kant moet de gebruiker bij gebruik van de CLI opdrachten invoeren om de taken uit te voeren. Over het algemeen is GUI gebruiksvriendelijker, maar de uitvoeringssnelheid is hoger in CLI.

Een CLI opdracht wordt dus uitgevoerd aan de hand van een commando. 

Nu is het mogelijk om een aantal extra commando’s te installeren die deel uitmaken van een overkoepelende tool. 

Dit hoofdstuk onderzoek de verschillende commando’s van de Flutter tool en vergelijkt deze met de verschillende Android tools en hun gelijkende commando’s.

\section{Opzet}
Een Software Development Kit (SDK) is een verzameling softwareontwikkelingstools in één installeerbaar pakket. Ze vergemakkelijken het maken van applicaties door een compiler, debugger en misschien een softwareframework te bevatten. Ze zijn normaal gesproken specifiek voor een combinatie van hardwareplatform en besturingssysteem. Een SDK voorziet een set tools, bibliotheken, relevante documentatie, codevoorbeelden, processen en/of handleidingen waarmee ontwikkelaars softwaretoepassingen op een specifiek platform kunnen maken. 


\section{Android}
De Android SDK bestaat uit meerdere tools die nodig zijn voor app-ontwikkeling. De tools kunnen geinstalleerd en bijgewerkt worden met SDK Manager van Android Studio of de command line tool ‘sdkmanager’. Alle tools worden gedownload naar de Android SDK-map. Aangezien de aard van het onderzoek een vergelijkende studie is werden hier enkel de commando’s aangehaald met een Flutter tegengestelde. Het aanhalen van alle commando’s voor elke tool zou te veel en niet interessant zijn.

\newpage

    \begin{tabular}{ |p{3cm}|p{11cm}|  }
        \hline
        \multicolumn{2}{|c|}{\textbf{Android SDK Command-Line Tools}} \\
        \multicolumn{2}{|c|}{(android\_sdk/cmdline-tools/version/bin/)} \\
        \hline
        apkanalyzer & Provides insight into the composition of your APK after the build process completes.\\
        \hline
        avdmanager & Allows you to create and manage Android Virtual Devices (AVDs) from the command line. \\
        \hline
        lint & A code scanning tool that can help you to identify and correct problems with the structural quality of your code. \\
        \hline
        retrace & For applications compiled by R8, retrace decodes an obfuscated stack trace that maps back to your original source code.\\
        \hline
        sdkmanager & Allows you to view, install, update, and uninstall packages for the Android SDK.\\
        \hline
    \end{tabular}

    \begin{tabular}{ |p{3cm}|p{11cm}|  }
    \hline
    \multicolumn{2}{|c|}{\textbf{Android SDK Build Tools}} \\
    \multicolumn{2}{|c|}{(android\_sdk/build-tools/version/)} \\
    \hline
    \multicolumn{2}{|c|}{This package is required to build Android apps. Most of the tools in here are invoked} \\
    \multicolumn{2}{|c|}{by the build tools and not intended for you.} \\
    \hline
    aapt2 & Parses, indexes, and compiles Android resources into a binary format that is optimized for the Android platform and packages the compiled resources into a single output.\\
    \hline
    apksigner & Signs APKs and checks whether APK signatures will be verified successfully on all platform versions that a given APK supports.\\
    \hline
    zipalign & Optimizes APK files by ensuring that all uncompressed data starts with a particular alignment relative to the start of the file.\\
    \hline
\end{tabular}


    \begin{tabular}{ |p{3cm}|p{11cm}|  }
    \hline
    \multicolumn{2}{|c|}{\textbf{Android SDK Command-Line Tools}} \\
    \multicolumn{2}{|c|}{(android\_sdk/cmdline-tools/version/bin/)} \\
    \hline
    apkanalyzer & Provides insight into the composition of your APK after the build process completes.\\
    \hline
    avdmanager & Allows you to create and manage Android Virtual Devices (AVDs) from the command line. \\
    \hline
    lint & A code scanning tool that can help you to identify and correct problems with the structural quality of your code. \\
    \hline
    retrace & For applications compiled by R8, retrace decodes an obfuscated stack trace that maps back to your original source code.\\
    \hline
    sdkmanager & Allows you to view, install, update, and uninstall packages for the Android SDK.\\
    \hline
\end{tabular}


\newpage

\section{Flutter}
De Flutter command-line tool is hoe ontwikkelaars (en IDE’s aangestuurd door ontwikkelaars) omgaan met Flutter. 

Voor het kunnen werken met Flutter moet dus eerst de Flutter SDK geïnstalleerd worden. Deze SDK bevat onder andere een aantal commando’s die kunnen uitgevoerd worden in de terminal. Flutter commando’s worden voorafgegaan door het flutter keyword. Zo weet het besturingssysteem dat een commando wordt gebruikt van de flutter tool.

%%pub get & dart commando's

Hieronder staat een lijst met de mogelijke flutter commando’s.

\section{Conclusie} 
%%=============================================================================
%% Appendix
%%=============================================================================

\chapter{\IfLanguageName{dutch}{Appendix}{Appendix}}
\label{ch:appendix}

\section{Hello world template}
De eerste Flutter app gemaakt in het kader van dit onderzoek was een welbekende en gebruikelijke Hello World applicatie. Een Hello World app klinkt alle ontwikkelaars bekend in de oren. Het is namelijk een standaard template die niet meer doet dan het tonen van een “Hello World” tekst op een scherm. Het is zeer minimaal in aantal lijnen code, wat ergens wel logisch is, en zorgt voor een heel kleine, compacte APK. Om enige vorm van realisme in de ontwikkeling van deze voorbeeld applicaties te integreren, zal er gekozen worden om de builds van deze applicaties via een release schema uit te voeren. Deze manier van builden optimaliseert de code in zijn geheel, zoals de applicatie terug te vinden zou zijn op de Play Store bijvoorbeeld.

\section{ABI}

\section{Opstart procedures van een applicatie}

Een cold start houdt in dat de applicatie vanaf nul opgestart wordt. Het systeem heeft namelijk, tot nu, nog niks gedaan om de applicatie zijn proces te starten. Cold starts komen voor wanneer het systeem net opgestart is of dat het systeem de applicatie net zelf afgesloten heeft (bijvoorbeeld omdat het systeem geheugen te kort had). Bij de cold start van een applicatie moet het systeem volgende stappen ondernemen:

\begin{enumerate}
    \item Laden en lanceren van de applicatie
    \item Tonen van een blanco start venster voor de applicatie direct na het lanceren ervan
    \item Creëren van het applicatie proces
\end{enumerate}

Van zodra het systeem klaar is met het creëren van het hierboven vermeld applicatie proces, is het de verantwoordelijkheid van de applicatie om de volgende stappen in zijn \emph{lifecycle} te ondernemen. Deze \emph{lifecycle} ligt echter buiten de scope van dit onderzoek. 

Een hot start is een andere vorm hoe de applicatie opgestart kan worden. Deze vorm is veel eenvoudiger en vergt minder werk als een cold start. Het enige wat het systeem bij een hot start dient te doen is de view naar de voorgrond brengen. Wanneer alle activiteiten van de applicatie nog in het geheugen aanwezig zijn, kan de applicatie enkele stappen in het opstart proces vermijden. Dit heeft als gevolgd dat het opstarten van de applicatie een pak sneller gaat. 

Een warm start dan ten slotte, is een combinatie van de cold en hot start. Er zijn verschillende toestanden die als warm start kunnen bekeken worden. Wanneer een gebruiker weg navigeert van de applicatie en vervolgens direct de applicatie opnieuw opstart, kan dit beschouwd worden als een warm start. Het proces kan mogelijks verder blijven lopen, maar de applicatie moet de view opnieuw creëren vanaf nul. Een ander voorbeeld van een warm start is wanneer het systeem de applicatie uit het geheugen verwijdert, maar de gebruiker de applicatie terug opstart. Het proces en de view moeten opnieuw gestart worden, maar het systeem kan enigszins profiteren van de opgeslagen instantiebundel. 

\section{Threads}
De dag van vandaag hebben smartphones meerdere processoren voor het uitvoeren van taken of processen. Deze processoren zijn in staat om verschillende taken gelijktijdig uit te voeren. Dit noemt men \emph{multi-processing}.

Om de processoren efficiënter te gebruiken kan het besturingssysteem een applicatie verplichten om meer dan één thread van uitvoering te creëren binnen een proces. Dit noemt men \emph{multi-threading}. Dit kan vergeleken worden met het lezen van meerdere boeken tegelijkertijd en wisselen tussen elk boek achter een hoofdstuk. In dit scenario is de lezer gelijk aan de processor, alle boeken samen zijn gelijk aan het uit te voeren proces, één boek gelijk aan een thread en het lezen van dat boek gelijk aan het uitvoeren van een thread. Het is echter niet mogelijk om in meerdere boeken tegelijkertijd te lezen.

Voor het beheren van deze threads is veel overhead nodig. Zo is er nood aan een \emph{Schedular}. Deze kijkt onder andere naar prioriteit van alle threads maar houdt ook rekening met het uitvoeren en finaliseren van al deze threads. Vervolgens is er de \emph{Dispatcher}, deze houdt zich bezig met het aanmaken van threads. In het voorbeeld kan dit vergeleken worden met een persoon die de boeken die gelezen dienen te worden bezorgd en een context aanbiedt waarin dit moet gebeuren. De context kan vergeleken worden met een speciale lees kamer. Sommige contexts zijn beter voor UI taken, andere beter voor I/O taken. 

Ook is interessant om weten dat gebruikers applicaties meestal een \emph{main thread} hebben. Deze wordt uitgevoerd in de voorgrond en kan andere threads dispatchen in de achtergrond.

%%=============================================================================
%% Conclusie
%%=============================================================================

\chapter{Conclusie}
\label{ch:conclusie}

% TODO: Trek een duidelijke conclusie, in de vorm van een antwoord op de
% onderzoeksvra(a)g(en). Wat was jouw bijdrage aan het onderzoeksdomein en
% hoe biedt dit meerwaarde aan het vakgebied/doelgroep? 
% Reflecteer kritisch over het resultaat. In Engelse teksten wordt deze sectie
% ``Discussion'' genoemd. Had je deze uitkomst verwacht? Zijn er zaken die nog
% niet duidelijk zijn?
% Heeft het onderzoek geleid tot nieuwe vragen die uitnodigen tot verder 
%onderzoek?

% Trek een duidelijke conclusie, in de vorm van een antwoord op de onderzoeksvra(a)g(en). Wat was jouw bijdrage aan het onderzoeksdomein en hoe biedt dit meerwaarde aan het vakgebied/doelgroep?

Al is het moeilijk om te laten uit blijken of Flutter al een volwaardige tegenstander is voor Android. Indien een grotere app ontwikkeld wordt met complexe features kan dit pas bekeken worden.

Voor simpele applicaties met weinig complexe features lijkt het daarom wel de beste optie. Echter wanneer een ontwikkelaar wilt groeien en bijleren zal hij al snel stoten op de werkelijkheid van het framework. De kennis over native Android en iOS hoeft echter niet uitgebreid te zijn. Maar Flutter vergt wel kennis over onderliggende Frameworks wanneer men uitgebreidere Flutter apps begint te ontwikkelen 


% Reflecteer kritisch over het resultaat. Had je deze uitkomst verwacht? Zijn er zaken die nog niet duidelijk zijn?

Het blijft belangrijk om de omvang van de applicaties in rekening te houden.
Ook zijn bepaalde delen van het onderzoek open voor discussie. Zo werden niet alle omgevingsvariabelen weg gehaald voor en tijdens het onderzoek. 

Complexe 

Deze conclusie van het onderzoek ligt in lijn met de hypotheses op de onderzoeksvragen. De literatuurstudie gaf duidelijk inzicht in het Flutter framework, hieruit kon een objectief beeld gevormd worden over de werking en de voor- en nadelen van de Flutter stack. 

% Heeft het onderzoek geleid tot nieuwe vragen die uitnodigen tot verder 
\textbf{Verder onderzoek}
Het is noodzakelijk dat de iOS 

%%=============================================================================
%% Bijlagen
%%=============================================================================

\appendix
\renewcommand{\chaptername}{Appendix}

%%---------- Onderzoeksvoorstel -----------------------------------------------

\chapter{Onderzoeksvoorstel}

Het onderwerp van deze bachelorproef is gebaseerd op een onderzoeksvoorstel dat vooraf werd beoordeeld door de promotor. Dat voorstel is opgenomen in deze bijlage.

% Verwijzing naar het bestand met de inhoud van het onderzoeksvoorstel
%---------- Inleiding ---------------------------------------------------------

\section{Introductie} % The \section*{} command stops section numbering
\label{sec:introductie}

De alom bekende smartphones zijn de dag van vandaag niet meer weg te denken en hoewel de markt verzadigd is, kunnen verschillende gelijkenissen getrokken worden tussen de toestellen. Allereerst maakt elk toestel gebruik van een besturingssysteem; voor Apple is dit iOS, terwijl Samsung het Android besturingssysteem gebruikt. Een ander aspect dat ze delen, is het gebruik van mobiele apps. Alvorens een ontwikkelaar begint met het schrijven van een app zal hij eerst moeten kiezen op welk(e) besturingssysteem/besturingssystemen deze app zal draaien. Uit deze keuze komen twee mogelijkheden voort.
\newline

Enerzijds native app ontwikkeling, dit zijn apps die rechtstreeks ontwikkeld worden voor een bepaald besturingssysteem. Anderzijds cross-platform app ontwikkeling, dit zijn apps die ontwikkeld en vervolgens gecompileerd kunnen worden voor meerdere besturingssystemen. Een aantal jaar geleden was dit een voor de hand liggende keuze, aangezien cross-platform ontwikkeling nog in zijn jonge schoentjes stond. Tegenwoordig wordt het aanzien als een kost besparend alternatief door het hergebruik van resources. Onder cross-platform ontwikkeling vinden we verschillende frameworks terug die kunnen gebruikt worden. De grootste zijn: Flutter, Xamarin, React Native en Ionic. Dit onderzoek zal zich toespitsen op Flutter, een jong ontwikkelingsplatform waar momenteel relatief weinig onderzoek naar verricht werd. Desondanks heeft het platform veel potentieel en een snelgroeiende gebruikersbasis. Dit onderzoek zal een beeld schetsen van de voor- en nadelen van beide platformen. Het softwareontwikkelingsbedrijf NextApps, wil met behulp van de resultaten van dit onderzoek een beter inzicht krijgen in de werking van cross-platform development met het Flutter framework. Dit onderzoek zal een antwoord proberen vormen op de volgende onderzoeksvragen: ‘Wat zijn de voor- en nadelen van app ontwikkeling in Flutter in vergelijking met native Android?’, ‘Is Flutter al matuur genoeg om te aanschouwen als volwaardig alternatief op native app ontwikkeling?’, ‘Is Flutter toegankelijker voor nieuwe ontwikkelaars?’.



%---------- Stand van zaken ---------------------------------------------------

\section{State-of-the-art}
\label{sec:state-of-the-art}

Flutter, ontwikkeld door Google, is een relatief jong framework waarin, volgens de site\footnote{\url{https://flutter.dev}}, mooie, native gecompileerde mobiele-, web- en desktopapplicaties kunnen ontwikkeld worden. Het platform werd aangekondigd door Google in 2015, maar was pas echt gangbaar in december 2018 toen de eerste stabiele versie uitkwam. Flutter is opensource, wat wil zeggen dat er vrije toegang is tot de bronmaterialen. Dit zorgt voor een verdere ontwikkeling van het framework en bevordert een hoog niveau van innovatie. De jonge aard van Flutter betekent anderzijds dat nog niet veel research over het framework verschenen is. De meeste papers en artikels vergelijken de verschillende cross-platform frameworks onderling om zo een beeld te schetsen van alle voor- en nadelen van elk platform. Dit helpt bij het kiezen van een cross-platform framework maar beantwoordt niet de vraag: zou het beter zou zijn om een app native te ontwikkelen?\newline

Het onderzoek zal gebruik maken van drie recente papers die soortgelijke onderzoeksvragen hadden. De eerst paper, van onderzoeker \textcite{Bracke2020}; vertoont verscheidene gelijkenissen met de onderzoeksvragen van deze paper en onderzocht Flutter tegenover native Android ontwikkeling. De tweede paper van \textcite{Olsson2020}; onderzoekt het Flutter framework tegenover native app ontwikkeling, wat betekend dat het onderzoek ook rekening houdt met iOS, wat deze paper niet zal doen. De laatste paper van \textcite{Cheon2020}; is een paper over de creatie van een Flutter app uit een reeds bestaande Android app. Voor de ontwikkeling van de Flutter app, zal de gids van \textcite{Payne2019} gebruik worden. Verder biedt de Flutter site\footnote{\url{https://flutter.dev/docs}} duidelijke en overzichtelijke documentatie aan.


% Voor literatuurverwijzingen zijn er twee belangrijke commando's:
% \autocite{KEY} => (Auteur, jaartal) Gebruik dit als de naam van de auteur
%   geen onderdeel is van de zin.
% \textcite{KEY} => Auteur (jaartal)  Gebruik dit als de auteursnaam wel een
%   functie heeft in de zin (bv. ``Uit onderzoek door Doll & Hill (1954) bleek
%   ...'')

%---------- Methodologie ------------------------------------------------------
\section{Methodologie}
\label{sec:methodologie}

De uitvoering van het experiment van dit onderzoek omvat het schrijven van twee applicaties. De eerste app zal geschreven worden in het Flutter framework, gebruikmakend van de Dart programmeertaal. De andere app zal geschreven worden in native Android, gebruikmakend van Kotlin\footnote{\url{https://kotlinlang.org}}. Als onderdeel van het uit te voeren experiment, zal het schrijven van beide applicaties gedocumenteerd worden. Het onderzoek zal een finale conclusie vormen op de onderzoeksvragen, gebaseerd op de analyse van de onderzoeksresultaten. Op basis van deze resultaten kan het Flutter platform worden beoordeeld. Ten slotte zal het onderzoek de voor- en nadelen van elk systeem oplijsten.\newline

Beide apps zullen ontwikkeld worden met de Android SDK. De twee apps zullen ook op een gelijkaardige manier ontworpen worden om de user experience tussen de twee apps zo gelijk mogelijk te houden. Als maatstaaf voor de ontwikkelde applicaties worden een aantal minimumvereiste opgelegd zoals een goede performantie en een mooie visuele samenhang. Bij het schrijven van de applicaties zullen verscheidene aspecten bekeken en vergeleken worden. Het onderzoek zal gebruik maken van een lijst met richtlijnen. Deze zullen vermelden wat de vooropgestelde functionaliteiten van de apps zullen zijn.\newline

De te vergelijken aspecten:
\begin{itemize}
	\item Grootte van de uitvoeringsbestanden en opstartsnelheid van de app
	\item CPU gebruik van de app
	\item Gebruik van online API’s
	\item Security
	\item Beschikbare libraries en code complexiteit
	\item Creatie van views
	\item Asynchroon werken
	\item Beschikbare tools
\end{itemize}

\section{Verwachte Resultaten}
\label{sec:verwachte_resultaten}

Uit eerdere onderzoeken blijkt Native app ontwikkeling een beter alternatief te bieden op vlak van performantie en opstartsnelheid. Een Flutter APK moet volgende zaken bevatten: de core engine, Java-, app- en framework code, het licentie bestand en ICU data. Een Flutter APK heeft een minimale grote van 4.7MB omdat het zijn eigen resources nodig heeft. Ook zal rekening gehouden worden met de leeftijd van het Flutter platform, hier worden nog wat imperfecties verwacht. Door het grote potentieel en de vele steun die Flutter heeft, zal dit naar de toekomst toe, hoogstwaarschijnlijk worden opgelost. Van Flutter wordt verwacht dat het toegankelijker is voor nieuwe applicatieontwikkelaars. Ook wordt verwacht dat Flutter de bovenhand zal nemen op vlak van code complexiteit en creatie van views.

\section{Verwachte conclusies}
\label{sec:verwachte_conclusies}

Op basis van de resultaten van het onderzoek zal een conclusie op de onderzoeksvragen gevormd worden. In essentie zal deze conclusie antwoorden op de vraag, is Flutter al ver genoeg ontwikkeld om aanschouwd te worden als volwaardige tegenhanger van native development? Gebruikmakend van de lijst met richtlijnen en de behaalde scores, zal hier hopelijk een duidelijk antwoord gegeven worden.



%%---------- Andere bijlagen --------------------------------------------------
%\input{...}

%%---------- Referentielijst --------------------------------------------------

\printbibliography[heading=bibintoc]

\end{document}
