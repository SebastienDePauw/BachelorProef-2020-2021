%%=============================================================================
%% Stand van zaken
%%=============================================================================


\chapter{\IfLanguageName{dutch}{Stand van zaken}{State of the art}}
\label{ch:stand-van-zaken}

% Tip: Begin elk hoofdstuk met een paragraaf inleiding die beschrijft hoe
% dit hoofdstuk past binnen het geheel van de bachelorproef. Geef in het
% bijzonder aan wat de link is met het vorige en volgende hoofdstuk.

% Pas na deze inleidende paragraaf komt de eerste sectiehoofding.

Dit hoofdstuk is equivalent aan een literatuurstudie. De eerste stap in dit onderzoek was het zoeken naar voorgaande studies over Flutter en Android. Elke studie die voldeed aan de eisen en als interessant werd beschouwd, werd opgeslagen en gebundeld. Hiervan werd de inhoud diagonaal gelezen om zo alleen de meest relevante onderzoeken over te houden.

Vervolgens werd een opsomming gemaakt en een kort beeld geschetst van een aantal van deze studies en hun bijdrage in het vakgebied. In de volgende sectie zullen dus alleen de meest relevante studies, die dicht aansluiten bij dit onderzoek, worden aangehaald. De ondervindingen uit deze studies werden gebruikt als voorbereiding op het onderzoek en kunnen in volgende hoofdstukken gebruikt worden als steunpunten. Door de ondervindingen van verschillende studies te bundelen, kon gezocht worden naar een rode draad.

De te onderzoeken criteria van deze studie werden vastgelegd op basis van de reeds voorgaande onderzoeken. Het is de bedoeling dat deze studie bijdraagt aan het vakgebied. Hierdoor wordt weerhouden van reeds onderzochte criteria opnieuw te gaan onderzoeken. Toch zal blijken dat sommige onderzochte criteria opnieuw worden bekeken. Deze herhaling komt voort uit de Flutter 2.0 release, deze zorgde voor veranderingen in de prestatiemogelijkheden van het framework.\newpage

\section{Voorgaande onderzoeken}
\label{sec:voorgaande-onderzoeken}
 Navaron Bracke onderzocht de verschillen en gelijkenissen tussen Android en Flutter in een vergelijkende studie. \autocite{Bracke2020}. De scriptie vormde een beeld over de toenmalige stand van het Flutter framework door het te toetsen tegen het ontwikkelde Android. Het onderzoek was afgebakend door een vooraf vastgelegd aantal criteria. Zo kon de onderzoeker zekerheid bieden over de grondigheid van de studie. De onderzochte criteria waren: Internationalisering, navigatie, persisteren van gegevens, user Interfaces, asynchroon werk, permissies, software testing, opstarttijd van de applicatie, grootte van de applicatie. De opzet en het doel van het onderzoek liep in grootte mate gelijk met het onderzoek gevoerd voor deze scriptie. Daarom was het onderzoek van Bracke uitermate interessant, bevindingen werden bijgehouden, genoteerd en getoetst tegen de bevindingen van dit onderzoek. De hoofdstukken die performantie onderzochten werden gebruikt om een beeld te schetsen van de Flutter evolutie over tijd. De andere twee hoofdstukken (User Interfaces en Asynchroon werk) werden opnieuw onderzocht met conclusies van Bracke in het achterhoofd. Hiervoor werd, in de mate van het mogelijke, zoveel mogelijk gefocust op de onderwerpen die Bracke niet aanhaalde. De onderzoeker constateerde dat, ondanks de jonge aard van het Flutter platform, het een goed alternatief biedt op elk onderzocht criterium. Bij elk van deze criteria werd een platform van voorkeur gekozen. Hieruit bleek Flutter op 4 van de 7 criteria naar voor te komen. Al werd wel meegedeeld dat niet alle features in Flutter even gebruiksvriendelijk zijn maar door de jonge aard werd verwacht dat dit in de toekomst zou verbeteren.

Daniel Dang en David Skelton, twee professors aan de Insitutue of Technology, schreven een wetenschappelijk artikel over mobile app development in het onderwijs \autocite{Dang2019}. Het doel van het onderzoek was het bepalen van het ideale framework voor het aanleren van mobiele applicatieontwikkeling in het onderwijs. Het onderzoek werd gevoerd aan de hand van verschillende native- en cross-platform frameworks. Zo werden native Android, native iOS, Flutter, React Native, Ionic, Xamarin, Cordova en Appcelerator vergeleken. Onderzochte criteria bij dit onderzoek waren: Design App User Interface, User Event Handling, Services \& Multiple threads used in Internet connection, Graphics and Animation, Device hardware and sensor, Wireless connectivity, Internal data storage, Real time database. Dit artikel werd gebruikt als basis voor het sommige delen van het onderzoek gevoerd in deze scriptie. Zo werden de hoofdstukken User Event Handling en Services \& Multiple threads used in Internet connection grondig gelezen en ontleed voor dit onderzoek. In conclusie bleken Flutter en Native Android de twee beste platformen te zijn voor gebruik tijdens praktijklessen. Ook beval het onderzoek acht ideale onderwerpen aan die in de praktijksessies moeten worden behandeld om studenten voldoende technische vaardigheden bij te brengen om alle soorten mobiele apps te ontwikkelen. Belangrijk hierbij te vermelden is dat de studie gebruik maakte van een oude Flutter release.

Austen Latture onderzocht de ontwikkelcyclus van een app in Flutter \autocite{Latture2020}. De focus van het onderzoek lag hem op de leercurve van het platform. Hier waren geen vooropgestelde criteria aanwezig, deze studie omschreef de uitwerking van de Backdrop app. In conclusie bleek Flutter een toegankelijk platform te zijn voor nieuwe ontwikkelaars met een allesbehalve steile leercurve. Een ander groot voordeel aangeboden door het platform is de talrijke ingebouwde functionaliteit. De Dart taal die wordt geschreven voor het maken van Flutter apps is volgens het onderzoek gemakkelijk aan te leren voor mensen met een achtergrond in de C taal. Aan de andere kant bleek het kiezen van third-party libraries een hinder blok te vormen tijdens de ontwikkelfase. De jonge aard van het platform ging gepaard met jonge libraries. Zoals bleek uit andere studies concludeerde dit onderzoek dat Flutter hier een rode draad in mist. De grootte hoeveelheid aangeboden libraries maakt het moeilijk om te weten welke goed ondersteund worden en dan ook vaak gekozen worden. Dit is vooral een struikelblok voor nieuwe developers, onbekend met het platform. Vervolgens werd Flutter ook in het kort vergeleken met React Native. Hieruit kon geconcludeerd worden dat Flutter veel gemakkelijker te installeren is. Dit komt, volgens het onderzoek, voort uit de ingebouwde functionaliteit van Flutter terwijl React meer werkt met verspreide derde partij software die meestal bestaat uit te veel boilerplate code. Het onderzoek van Latture hielp dit onderzoek met antwoorden op één van de deelonderzoeksvragen 'Is Flutter toegankelijk voor nieuwe ontwikkelaars?'. Ondanks de subjectiviteit van de paper, werd wel rekening gehouden met de bevindingen van het onderzoek. Echter kon niet naast de onwetenschappelijke aard van het onderzoek gekeken worden, hierdoor wordt verder niet gerefereerd naar deze bevindingen.

Carlos Chavez, een student en Yoonsik Cheon, een professor aan de Universiteit van Texas El Paso onderzochten het herschrijven van Native Android apps naar Flutter apps \autocite{Cheon2020}. Het onderzoek bekeek een native Android app ontwikkeld in de Java taal en probeerde deze zo goed mogelijk om te zetten naar een Flutter app die terug kon gecompileerd worden als een Android app maar ook als iOS app. Het onderzoek werd gevoerd aan de hand van het Flutter ontwikkelproces. De technische-, praktische problemen en uitdagingen die de kop opstaken werden vervolgens besproken. Het onderzoek deed een paar duidelijke verschillen uit blijken tussen Android en Flutter wanneer gekeken werd naar de taal, programmeerstijl en design. Het grootste werk was het herschrijven van de views terwijl het herschrijven van de achterliggende logica eerder snel gebeurd was. Het onderzoek focuste zich onder andere op het aantal lijnen code geschreven voor beide applicaties. De Flutter code bleek hierbij maar 2/3 van het totaal aantal Android code te zijn. Maar volgens de onderzoekers was dit niet genoeg om apps in Flutter te beginnen ontwikkelen. De leeftijd van het platform bracht nog te veel onzekerheid met zich mee. Ook ontbrak dit onderzoek aan een soort standaard werkwijze of verzameling van richtlijnen. Dit sloeg dan vooral op libraries en API’s binnen het platform. De gebruikersbasis en de beschikbare tools moesten hierbij ook zeker in rekening worden gebracht. Het grote aantal third party libraries, door de grootte aanhang, zorgde voor onzekerheid bij het zoeken naar een goed ondersteunde library. Bij andere grootte platformen wordt dit verholpen doordat na bepaalde duur de beste API’s en libraries naar boven komen en een soort van community standard wordt gezet. Dit onderzoek was uitermate interessant en legde gronde voor een groot deel van dit onderzoek. Ook al zijn geen raakvlakken aanwezig tussen de twee onderzoeken, veel kennis vergaard uit deze paper werd gebruikt als bouwsteen van dit onderzoek.

Mathilda Olsson onderzocht hoe Flutter applicaties vergelijken met native applicaties \autocite{Olsson2020}. Opnieuw een onderzoek dat dicht aansluit bij het onderzoek gevoerd voor deze scriptie. Voor haar onderzoek ontwikkelde Olsson twee native applicaties, één in Kotlin Android en één in Swift iOS. Deze applicaties werden vervolgens beoordeelt op vlak van CPU performantie. Uit onderzoek bleek dat Flutter en Android apps gelijkaardige resultaten halen op vlak van CPU performantie. Vervolgens werd gekeken naar het aantal lijnen code en de complexiteit van de twee code bases. Hieruit bleek dat Flutter aanzienlijk minder lijnen code nodig had maar ook minder complexe code bevatte. De Flutter app bestond uit 125 lijnen code terwijl de twee native apps samen 580 lijnen code bevatte. Dit verschil is substantieel en duidt erop dat Flutter beter blijkt voor kleine tot middelgrote applicaties. De bevindingen over CPU performantie werden gebruikt in dit onderzoek voor het schetsen van een beeld van CPU performantie over tijd. Daarbuiten bleek Flutter opnieuw een betere optie over Android, wat leidde tot het vormen van een hypothese op de onderzoeksvraag. Het onderzoek van Olsson bevatte een bevraging die zich focuste op de verschillen in gebruikersperceptie. Een aantal gebruikers moesten de app eerst testen waarna ze een vragenlijst konden beantwoorden waaruit verschillen in look en feel moesten duidelijk worden. Uit de vragenlijst bleek native ontwikkeling, op vlak van UI, de voorkeur te hebben van het grootste deel gebruikers. De verschillen waren vooral te zien bij animaties, fonts, gedrag, lijsten en menu’s. Animaties konden wel gelijkgesteld worden maar dat is extra werk zijn voor de ontwikkelaar. In conclusie, zoals in de andere onderzoeken, werd nogmaals geduid op de leeftijd van het platform. Alle criteria waar Flutter minder op scoorde konden nog volop in ontwikkeling zitten voor een volgende release. Flutter had volgens het onderzoek veel potentieel indien de steun van de gebruikersbasis groot bleef.

Jakhongir Fayzullaev onderzocht drie verschillende cross-platform development frameworks, Kotlin Native, Multi-OS en Flutter in een vergelijkende studie \autocite{Fayzullaev2018}. De frameworks en dus de studie is vooral interessant voor Android ontwikkelaars aangezien de onderzochte frameworks dicht aansluiten bij de Java en Android ontwikkeling methoden. In 2018, toen het onderzoek werd uitgevoerd, waren de Flutter en Multi-OS nieuwe spelers op de cross-platform markt. Het onderzoek kon zich niet baseren op voorgaande studies en ontbrak aan duidelijke richtlijnen van de frameworks. Hierdoor was het onderzoek oppervlakkig, het vergeleek de grote lijnen van de platformen zonder de details te bekijken. Het Flutter deel van het onderzoek bekeek in grote lijnen: Widgets, het maken van layouts en de bouwstenen van Flutter. Fayzullaev concludeerde dat op vlak van performantie de drie platformen zo goed als identiek waren. Onderzoek deed blijken dat Flutter een goed platform was in 2018 maar de jonge aard zorgde voor weinig bruikbare libraries en tools. Ook had Flutter weinig built-in functionaliteit met als gevolg meer werk voor de programmeurs die meer code moesten schrijven ten opzichte van de andere meer ontwikkelde platformen. Fayzullaev omschrijft deze manier van coderen als ‘het opnieuw uitvinden van het wiel’. Ondanks het ontbreken van detail in dit onderzoek werd het wel gebruikt om kennis op te doen over de bouwstenen van Flutter.

Sebastian Faust schreef een Bachelorproef over het maken van grootschalige Flutter applicaties \autocite{Faust2020}. Het doel van dit onderzoek is andere applicatieontwikkelaars helpen wanneer zij voor dezelfde problemen komen te staan tijdens het schrijven van grote Flutter apps. Veel van de andere onderzoeken concludeerden dat Flutter beter is voor kleinschalige applicaties. Daarom was dit onderzoek interessant om op te nemen in deze literatuurstudie. Het vormde een benchmark onderzoek voor de schaalbaarheid van Flutter applicaties. Voor dit onderzoek schreef Faust een app met veel complexe features, tijdens dit proces documenteerde hij alle design keuzes en de problemen/obstakels die hieruit voortkwamen. Na het evalueren van elk obstakel kon hij altijd tot een optimale oplossing komen. Na de studie werden ondervindingen gebundeld en een set van richtlijnen opgesteld voor het ontwikkelen van grote Flutter applicaties. Richtlijnen en best practices waar developers zich best aan houden tijdens het ontwikkelingsproces. Deze werden gebundeld in een gids die later werd gepubliceerd en goed ontvangen werd door de Flutter community. Voor verdere ondersteuning van het onderzoek werd nog een interview afgenomen met een Flutter expert. De conclusie van de paper toonde de goede schaalbaarheid van Flutter apps aan maar dit onderzoek werd vooral gebruikt voor het oplossen van hinderblokken die Faust gedetailleerd onderzocht

Als laatste is het interessant om kort de nieuwe Flutter update aan te halen \autocite{Flutter2021}. De impact hiervan op dit onderzoek en de verschillen met vorige studies te duiden. Op woensdag 3 maart 2021 werd Flutter Engage georganiseerd, een live event gebracht door het Flutter team. Niet alleen het Flutter platform kreeg een upgrade maar ook de taal Dart werd verbeterd voor een vlottere en vertrouwelijke programmeer ervaring. De grootste Flutter upgrade was de overgang van een mobiel framework naar een portable framework. Oorspronkelijk werd Flutter alleen gecompileerd voor iOS en Android besturingssystemen maar door de upgrade zijn daar: Windows, macOS, Linux en Web bijgekomen. Echter valt dit buiten de scope van dit onderzoek. De grootste Dart upgrades worden hieronder opgelijst: 
\begin{itemize}
    \item Sound null safety is toegevoegd, zonder oude code te breken. Dart zal zeggen waar mogelijks gevaren zitten op null excepties.
    \item Smarter flow analysis
    \item Late variables
    \item Required named parameters
    \item DevTools upgrade
\end{itemize}

Ook werden in de keynote verschillende grote bedrijven zoals Ubuntu, Microsoft, Toyota… genoemd die in de toekomst nauw gaan samenwerken met Flutter. Deze samenwerkingen tonen de kracht, potentieel en groei van het platform. Ook werd in de Keynote het aantal open issues op Github aangehaald. Doordat het een opensource framework is kan iedereen die de broncode wil zien, deze gaan opzoeken op GitHub. Moest een fout stuk code, een simpeler te schrijven stuk code of zelfs als een goede bijdrage aan het platform gevonden worden kan hier een issue van gemaakt worden. Als Flutter deze verbetering heeft nagekeken en goedgekeurd wordt deze toegevoegd aan de broncode. Op deze manier sparen zichzelf werk uit en groeit het platform snel en volgens de wensen van de gebruikers. Het grootte aantal issues die momenteel open staan zien zij dan ook als een teken van groei. De vele gebruikers die Flutter willen beter maken is een teken dat er veel vertrouwen is in het platform. Door de updates en aanpassingen zullen veel API’s verouderd zijn. Flutter heeft dit echter goed geanticipeerd door het flutter fix commando toe te voegen. Deze zal de bestaande code doorlopen en tonen welke API’s verouderd zijn en in wat ze best vervangen worden.

\section{Waarom deze studie?}
\label{sec:waarom-deze-studie}
Cross-platform development heeft een snelgroeiende gebruikersbasis. De vele updates en verbeteringen aan de platformen maakt cross-platform development dan ook elke dag aantrekkelijker. Dit leidt tot meer onderzoek en studies binnen de sector. Tijdens de literatuurstudie bleek dat een groot aantal papers Flutter vergeleek met andere cross-platform ontwikkel tools. Dit soort papers moet de lezer helpen bij het maken van een keuze tussen deze tools. Door het verdelen van de aandacht over verschillende platformen, komt Flutter niet uitgebreid aanbod in dit soort papers. Voor 2020 waren de meeste onderzoeken van deze aard. Flutter leek toen nog niet stabiel en groot genoeg om als afzonderlijke tool onderzocht te worden.

Zoals uit voorgaande sectie blijkt, is Flutter meerdere malen onderzocht in 2020. In verschillende studies werd het tekort aan onderzoek voor 2020 aangehaald. Flutter is in het afgelopen jaar in een stroomversnelling geraakt op vlak van onderzoek. Dit is te danken aan de sterke stijging in gebruikersaantal. Het platform is nog steeds relatief jong maar de aantrekkingskracht ligt hem in de gebruikersbasis en de grote steun. Hoe meer gebruikers, hoe belangrijker het product in kwestie. Hoe belangrijker het product, hoe meer middelen hierin worden geïnvesteerd en dat lokt op zijn beurt weer meer gebruikers. Deze cirkel wordt alleen onderbroken als het platform in kwestie zijn aantrekkelijkheid verliest. Flutter lijkt hier geen last van te hebben en deze studie wil laten blijken waarom.

Dus kan geconcludeerd worden dat Flutter een interessant onderzoeksdomein is. Maar zoals bij elk onderzoek binnen de IT, zijn bevindingen al snel verouderd. De rappe evolutie binnen de sector doet elk product streven naar de beste versie van zichzelf. De competitie is hoog, dus moet performantie altijd voorop staan. Bij elke update of verbetering zijn voorgaande performantie studies verouderd en niet accuraat. Hierdoor zijn de studies, aangehaald in dit hoofdstuk, dan ook vaak zo recent mogelijk. Toch onderging Flutter net een grote release. Vele zaken werden verbeterd, onder andere de performantie van het platform. De voorgaande studies zijn weer verouderd en hier ligt een kans om opnieuw onderzoek te doen en beeld te schetsen van de huidige performantie van Flutter. Ook lijkt het interessant om een beeld te schetsen van de performantie van Flutter over tijd.
